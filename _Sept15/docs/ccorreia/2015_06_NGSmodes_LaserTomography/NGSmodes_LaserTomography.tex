%%%%%%%%%%%%%%%%%%%%%%%%%%%%%%%%%%%%%%%%%%%%%%%%%%%%%%%%%%%%%%%%%%%%%%%%%%%%%%%
%                         File: osa-revtex4-1.tex                             %
%                        Date: April 15, 2013                                 %
%                                                                             %
%                              BETA VERSION!                                  %
%                   JOSA A, JOSA B, Applied Optics, Optics Letters            %
%                                                                             %
%            This file requires the substyle file osajnl4-1.rtx,              %
%                   running under REVTeX 4.1 and LaTeX 2e                     %
%                                                                             %
%                   USE THE FOLLOWING REVTeX 4-1 OPTIONS:                     %
% \documentclass[osajnl,twocolumn,showpacs,superscriptaddress,10pt]{revtex4-1}%
%                    %% Use 11pt for Applied Optics                           %
%                                                                             %
%               (c) 2013 The Optical Society of America                       %
%                                                                             %
%%%%%%%%%%%%%%%%%%%%%%%%%%%%%%%%%%%%%%%%%%%%%%%%%%%%%%%%%%%%%%%%%%%%%%%%%%%%%%%

%\documentclass[osajnl,twocolumn,showpacs,superscriptaddress,11pt]{revtex4-1} %% use 11pt for Applied Optics
%\documentclass[osajnl,preprint,showpacs,superscriptaddress,11pt]{revtex4-1} %% use 12pt for preprint option
\documentclass[a4paper,12pt]{article}
\usepackage{graphicx}
%\usepackage[ruled,vlined]{algorithm2e}
\usepackage{amsmath,amssymb,amsfonts}       % Typical maths resource packages

\usepackage{color}
%\usepackage{hyperref}%
\definecolor{gris30}{gray}{0.40}
\definecolor{orange}{rgb}{1,0.5,0}
\definecolor{mygreen}{rgb}{0,0.4,0}
\usepackage[bookmarksopen=true,bookmarksnumbered=true,breaklinks=true,linktocpage=true,colorlinks=true,linkcolor=orange,citecolor=mygreen,pdfusetitle,
               pdfauthor={Carlos Correia},pdftitle={GPI TT controller},
               pdfkeywords={LQG control, Kalman filtering, Adaptive
                 Optics, Gemini Planet Imager}]{hyperref}%
\usepackage[hyperpageref]{backref} % back references in biblio

%\usepackage{mathrsfs}
%\usepackage{enumerate, enumitem}

%\usepackage{multirow}
%\usepackage{arydshln}

%\usepackage{draftwatermark}
%\SetWatermarkScale{1}
%\usepackage[none, bottom,dark,outline,timestamp]{draftcopy}
%\usepackage{draftcopy}
%\draftcopyName{draft, v. 3.0, 04/06/2013 - to be submitted to JOSAA}{110}
%\draftcopyVersion{June 13}
%\draftcopySetScaleFactor{0.9}

\newcommand{\egref}[1]{\stackrel{\textrm{(\ref{#1})}}{=} } %          \stackrel{\textrm{(\ref{eq:phi_cor_double_stage})}}{=}
\newcommand{\egrefs}[2]{\stackrel{\textrm{(\ref{#1},\ref{#2})}}{=} }
\newcommand{\refeq}[1]{(eq.~\ref{#1})}

\newcommand{\ccc}[1]{{\color{red}\textrm{#1}}}

\def\mathbi#1{\textbf{\em #1}}

% - hyphenation penalties ---
\hyphenpenalty=50000
\tolerance=10000

\newcommand{\tildesi}{\tilde{s^i}}
\newcommand{\ipif}{2j\pi f}
\newcommand{\tildehu}{\tilde{h_1^i}}
\newcommand{\tildehd}{\tilde{h_2^i}}
\newcommand{\eipif}{e^{j\bk_i f}}
\newcommand{\eipifm}{e^{-j\bk_i f}}
\newcommand{\absd}{\right |}
\newcommand{\absg}{\left |}
\newcommand{\fleche}{$\Longrightarrow$}
%\newcommand{\argmax}{\mathop{\mathrm{arg\,max}}}\
\newcommand{\argmin}{\mathop{\mathrm{arg\,min}}}
\newcommand{\TFd}{\mathop{\mathbf{TF}}}
\newcommand{\TFc}{\boldsymbol{\mathcal{F}}} % Transformée de Fourier continue
%\newcommand{\Fc}{\boldsymbol{\mathcal{F}}} % Transformée de Fourier continue
\newcommand{\norm}[1]{\left \| {#1} \right \|}
\newcommand{\average}[1]{\left \langle {#1} \right \rangle}
\newcommand{\trace}[1]{\text{trace}\left \{ {#1} \right \}}
\newcommand{\bb}{{\bf b}}
\newcommand{\br}{{\bf r}}
\newcommand{\bff}{{\bf f}}
\newcommand{\bk}{{\bf k}}
\newcommand{\Cn}{$C_n^2\ $}
\newcommand{\Cnh}{$C_n^2(h)\ $}
\newcommand{\dro}{$\frac{D}{r_0} \ $}
\newcommand{\ro}{$r_0$}
%\newcommand{\Lo}{$L_0$}
\newcommand{\Lo}{$\mathcal{L}_0$}
\newcommand{\Da}{\text{D}_{ani}}
\newcommand{\laa}{\left \langle}
\newcommand{\raa}{\right \rangle}
\newcommand{\pho}{\Phi_{\text{res},0}}
\newcommand{\pha}{\Phi_{\text{res},\alpha}}
\newcommand{\lc}{\left [}
\newcommand{\rc}{\right ]}
\newcommand{\lp}{\left \lbrace}
\newcommand{\rp}{\right \rbrace}
\newcommand{\lpp}{\left (}
\newcommand{\rpp}{\right )}
\newcommand{\barred}{\right |}
\newcommand{\barreg}{\left |}
\newcommand{\alphab}{\text{\boldmath$\alpha$}}
\newcommand{\rhob}{\text{\boldmath$\rho$}}
\newcommand{\FEP}{\operatorname{FEP}}
\newcommand{\AIRY}{\operatorname{Airy}}
\newcommand{\FTO}{\operatorname{FTO}}
\newcommand{\SR}{\operatorname{SR}}
\newcommand{\Ec}{\operatorname{Ec}}
\newcommand{\CC}{\text{\large{\bf C}}}
\newcommand{\var}{\text{VAR}}
%\newcommand{\C}{\text{\bf{\large C}}}
\newcommand{\E}{\text{\bf{\large E}}}
\newcommand{\M}{\text{\bf{\large M}}}
\newcommand{\II}{\text{\bf{\large I}}}
\newcommand{\Ztr}{\text{\bf{\large Z}}}
\newcommand{\PP}{\text{\bf{\large P}}}
\newcommand{\pp}{\text{\bf p}}
\newcommand{\LL}{\text{\bf{\large L}}}
\newcommand{\R}{\text{{\large R}}}
\newcommand{\A}{\text{\bf{\large A}}}
\newcommand{\W}{\text{\bf{\large W}}}
\newcommand{\V}{\text{\bf{\large V}}}
\newcommand{\U}{\text{\bf{\large U}}}
\newcommand{\C}{\text{\bf{\large C}}}
\newcommand{\cc}{\text{\bf c}}
%\newcommand{\D}{\text{\bf{\large D}}}
\newcommand{\B}{\text{\bf{\large B}}}
\newcommand{\HH}{\text{\bf{\large H}}}
\newcommand{\OO}{\text{\bf{\large 0}}}
\newcommand{\epsB}{\text{\large{\boldmath$\epsilon$}}}
\newcommand{\PhiB}{\text{\boldmath$\Phi$}}
\newcommand{\varphiB}{\text{\large{\boldmath$\varphi$}}}
\newcommand{\varphib}{\text{{\boldmath$\varphi$}}}
\newcommand{\T}{\mathsf{T}} % matrix transpose
\newcommand{\Hm}{\text{H}} % matrix Hermitian transpose
\newcommand{\0}{\mathsf{0}} % null matrix 
\newcommand{\I}{\mathbf{I}} % identity matrix 
% \newcommand{\Ad}{\mathcal{A}_d} % transition matrix 
% \newcommand{\Bd}{\mathcal{B}_d} % input matrix 
% \newcommand{\Cd}{\mathcal{C}_d} % observation matrix 
% \newcommand{\Dd}{\mathcal{D}_d} % input matrix 
\newcommand{\Hkal}{\mathcal{H}_{\infty}} % Kalman gain
\newcommand{\Lkal}{\mathcal{L}_{\infty}} % Kalman gain
\newcommand{\K}{\mathcal{K}_{\infty}} % LQG gain
\newcommand{\SigmaK}{\Sigma_\infty} % LQG gain
\newcommand{\Fc}{\boldsymbol{\mathcal{F}}} % Transformée de Fourier continue
\newcommand{\TF}{\text{\bf TF}} % Transformée de Fourier continue
\newcommand{\z}{{z}} % Z-transform integrand
%\newcommand{\N}{\text{\bf N}} % DM influence function operator


\newcommand{\Proj}{\mathbf{H}} % Projector
\newcommand{\Rec}{\mathbf{R}} % reconstructor
\newcommand{\Emv}{\mathbf{E}} % Estimator, minimum-variance
\newcommand{\Fit}{\mathbf{F} }% Fitting operator
\newcommand{\CovMat}{\boldsymbol{\Sigma}} % reconstructor

% --- AR model matrices --
\newcommand{\Aar}{\mathbf{A}} % AR model A matrix
\newcommand{\Barr}{\mathbf{B}} % AR model B matrix
\newcommand{\Car}{\mathbf{C}} % AR model C matrix


\newcommand{\TFD}{\text{\bf D}} % TF WFS  operator
\newcommand{\TFN}{\text{\bf N}} % TF DM influence function operator
\newcommand{\DSP}{\text{\bf W}} % DSP d'une fonction

% ---- AO ----
\newcommand{\N}{\mathbf{N}} % DM influence function operator
\newcommand{\D}{\mathbf{G}} % WFS  operator
\newcommand{\Mcom}{\mathsf{M}_{com}} % matrice commande
\newcommand{\Mint}{\mathsf{M}_{int}} % matrice d'interaction
\newcommand{\mmse}{\text{\tiny{MMSE}}} % mmse reconstructor
\newcommand{\LS}{\text{\tiny{LS}}} % LS reconstructor
\newcommand{\xz}{\text{\bf X}} % Z-transform of state x
\newcommand{\uz}{\text{\bf U}} % Z-transform of command u
\newcommand{\yz}{\text{\bf S}} % Z-transform of measurements y
\newcommand{\WT}{\text{wt}} % Woofer-Tweeter acronym
\newcommand{\woo}{\text{w}} 
\newcommand{\twe}{\text{t}} 
\newcommand{\tr}{\operatorname{trace}} % Z-transform of measurements y 
\newcommand{\eig}{\operatorname{;}}
\newcommand{\ImSpace}{\operatorname{Im}}
%\newcommand{\var}{\operatorname{var}}}
%\newcommand{\ln}{\operatorname{ln}}
\newcommand{\Esp}{\operatorname{E}}
\newcommand{\lemmaend}{\hfill \ensuremath{\Box}}
\newcommand{\remend}{\hfill \ensuremath{\blacktriangleleft}}
\newcommand{\exend}{\hfill \ensuremath{\blacktriangleleft}}
\newcommand{\thend}{\hfill \ensuremath{\blacktriangle}}
\newcommand{\propend}{\hfill \ensuremath{\bigtriangleup}}
\newcommand{\tur}{\mathsf{\tiny{tur}}} 
\newcommand{\m}{\mathsf{m}} 
\newcommand{\dint}{\mathrm{d}} % var intégration 
\newcommand{\res}{\mathsf{\tiny{res}}} 
\newcommand{\cor}{\mathsf{\tiny{cor}}} 
\newcommand{\crit}{\mathsf{\tiny{crit}}}
\newcommand{\opt}{\infty} 
\newcommand{\IC}{\mathsf{InfC}} 
\newcommand{\IInc}{\mathsf{InfI}} 
\newcommand{\bnu}{\vec{\nu}} 

\newcommand{\vib}{\mathsf{\tiny{vib}}}
\newcommand{\TT}{\mathsf{\tiny{TT}}}
\newcommand{\PS}{\mathsf{\tiny{PS}}}
\newcommand{\QPS}{\mathsf{\tiny{PS2Q}}}

\newcommand{\ps}{\chi}

\newcommand{\Ad}{\mathcal{A}_\mathsf{tur}} % transition matrix 
\newcommand{\Bd}{\mathcal{B}_\mathsf{tur}} % input matrix 
\newcommand{\Cd}{\mathcal{C}_\mathsf{tur}} % observation matrix 
\newcommand{\Dd}{\mathcal{D}_\mathsf{tur}} % input matrix 
\newcommand{\Vd}{\boldsymbol{\Gamma}} 

\newcommand{\oloop}{\mathsf{\tiny{ol}}} 
\newcommand{\sys}{\mathsf{\tiny{sys}}} 
\newcommand{\lead}{\mathsf{\tiny{lead}}} 
\newcommand{\lag}{\mathsf{\tiny{lag}}} 
\newcommand{\wfs}{\mathsf{\tiny{wfs}}} 
\newcommand{\dac}{\mathsf{\tiny{dac}}} 
\newcommand{\intgr}{\mathsf{\tiny{int}}} 

%\newcommand{\ol}{\mathsf{\tiny{ol}}} 
\newcommand{\cl}{\mathsf{\tiny{cl}}} 

\newcommand{\thetavec}{{\boldsymbol{\theta}}} 
\newcommand{\alphavec}{{\boldsymbol{\alpha}}} 
\newcommand{\betavec}{{\boldsymbol{\beta}}} 
\newcommand{\rhovec}{\boldsymbol{\rho}}%{{\boldsymbol{\rho}}} 
\newcommand{\etavec}{{\boldsymbol{\eta}}} 
\newcommand{\phivec}{{\boldsymbol{\psi}}} 
\newcommand{\varphivec}{{\boldsymbol{\varphi}}} 
\newcommand{\varepsilonvec}{{\boldsymbol{\varepsilon}}} 
\newcommand{\svec}{{\mathbf{s}}} 
\newcommand{\xvec}{{\mathbf{x}}} 
\newcommand{\vvec}{{\mathbf{v}}} 
\newcommand{\uvec}{{\mathbf{u}}} 
\newcommand{\rvec}{{\mathbf{r}}} 
\newcommand{\Asa}{{\mathbf{A}}}
\newcommand{\Aexp}{{\mathbf{A}_\text{\tiny{LAY}}}}
\newcommand{\AexpT}{{\mathbf{A}_\text{\tiny{LAY}}^\T}}

\newcommand{\Mexp}{{\mathcal{M}_\infty^\text{\tiny{LAY}}}}
\newcommand{\Lexp}{{\mathcal{L}_\infty^\text{\tiny{LAY}}}}
\newcommand{\Sigmaexp}{{\CovMat_\infty^\text{\tiny{LAY}}}}
\newcommand{\SigmaexpN}{{\CovMat_n^\text{\tiny{LAY}}}}
\newcommand{\Msa}{{\mathcal{M}_\infty}}
\newcommand{\SigmaSA}{{\CovMat_\infty}}
\newcommand{\SigmaSAN}{{\CovMat_n}}

\newcommand{\Ndm}{{\mathsf{N_\text{DM}}}} 
\newcommand{\Nact}{{\mathsf{N_\text{act}}}} 
\newcommand{\Nwfs}{{\mathsf{N_\text{WFS}}}} 
\newcommand{\Nslopes}{{\mathsf{N_\text{s}}}} 
\newcommand{\Nl}{{\mathsf{N_\text{l}}}} 
% \newenvironment{prop}[1][Proposition :]{\begin{trivlist}
% \item[\hskip \labelsep {\bfseries #1}]}{\end{trivlist}}
% \newenvironment{proof}[1][Proof :]{\begin{trivlist}
% \item[\hskip \labelsep {\bfseries #1}]}{\end{trivlist}}

%  \newenvironment{comment}[1][Comment:]{\begin{trivlist}
%  \item[\hskip \labelsep {\bfseries #1}]}{\end{trivlist}}

% \newcommand{\qed}{\hfill \ensuremath{\blacksquare}}



\newcommand{\ps}{\mu}
\newcommand{\ProjTheta}{\mathbf{P}_{\theta}\,} % angular projection matrix 
\newcommand{\Proj}{\mathbf{P}} % projection matrix 
\newcommand{\ProjPS}{\mathbf{P}_\text{\tiny{NGS}}} % projection matrix NGS2Z



\newtheorem{theorem}{Theorem}[section]
\newtheorem{lemma}[theorem]{Lemma}
\newtheorem{proposition}[theorem]{Proposition}
\newtheorem{corollary}[theorem]{Corollary}

\newenvironment{proof}[1][Proof]{\begin{trivlist}
\item[\hskip \labelsep {\bfseries #1}]}{\end{trivlist}}
\newenvironment{definition}[1][Definition]{\begin{trivlist}
\item[\hskip \labelsep {\bfseries #1}]}{\end{trivlist}}
\newenvironment{example}[1][Example]{\begin{trivlist}
\item[\hskip \labelsep {\bfseries #1}]}{\end{trivlist}}
\newenvironment{remark}[1][Remark]{\begin{trivlist}
\item[\hskip \labelsep {\bfseries #1}]}{\end{trivlist}}

\newcommand{\qed}{\nobreak \ifvmode \relax \else
      \ifdim\lastskip<1.5em \hskip-\lastskip
      \hskip1.5em plus0em minus0.5em \fi \nobreak
      \vrule height0.75em width0.5em depth0.25em\fi}

%%%%%%%%%%%%%%%%%% title page information %%%%%%%%%%%%%%%%%%
\title{NGS modes in laser tomography AO systems} 
%\author{C. Correia and K. Jackson}
\author{Carlos M. Correia}%\email{ccorreia@astro.up.pt} %% email address is required
% \affiliation{Institute of Astrophysics and Space Sciences, University of Porto, CAUP, Rua das Estrelas, 4150-762 Porto, Portugal}
% \affiliation{Dept. of Physics and Astronomy, Faculty of Sciences, University of Porto,
% Rua do Campo Alegre 687, PT4169-007 Porto, Portugal}
% %\affiliation{Centre for Astrophysics, University of Porto, Rua das Estrelas, 4150-762 Porto, Portugal}
% \affiliation{Adaptive Optics Laboratory, University of Victoria, Victoria, BC V8P 5C2}
% \author{Kate Jackson$^{3}$}
% \affiliation{Dept. of Mechanical Engineering, University of Victoria, Canada}
% \author{ and the Raven Team}
%\author{(C$^2$)\,$^{a,b,*}$, Kate Jackson$^b$, Jean-Pierre V\'eran$^a$ \textit{et al}}
% \affiliation{$^{a}$ Centre for Astrophysics, University of Porto, Rua das Estrelas, 4150-762 Porto, Portugal}
% \affiliation{$^{b}$Adaptive Optics Laboratory, University of Victoria, Victoria, BC V8P 5C2}
% \address{$^{c}$Herzberg Institute of Astrophysics, National Research Council, Canada}
%\\$^{*}$carlos.correia@nrc.gc.ca}
 
%\address{ $^{b}$AO lab, UVic, Victoria, BC Canada} 
%\email{$^*$ccorreia@astro.up.pt} %% email address is required

%\homepage{http://astro.up.pt/~ccorreia/} %% author's URL, if desired





%%%%%%%%%%%%%%%%%%% abstract and OCIS codes %%%%%%%%%%%%%%%%
%% [use \begin{abstract*}...\end{abstract*} if exempt from copyright]

%%%%%%%%%%%%%%%%%%%%%%% begin %%%%%%%%%%%%%%%%%%%%%%%%%%%%%%
\begin{document}
\maketitle 

\begin{abstract}
This note reviews the NGS modes modelling for laser tomography AO
systems. Three models are presented: \\
\textit{i)} tilt-tomography using a
combination of tilt and high-altitude quadratic modes that produce
pure tilt through cone projected ray-tracing through the wave-front
profiles; \\
\textit{ii)} a spatio-angular MMSE tilt estimation
anywhere in the field that is more general. \\
\textit{iii)} the
(straightforward) generalisation to dynamic controllers using
near-Markovian time-progression models from \cite{correia15}. 

These controllers will be used for the HARMONI NGS modes.
\end{abstract}

\vspace{50pt}
\newpage
\tableofcontents

%\ocis{(000.0000) General.} % REPLACE WITH CORRECT OCIS CODES FOR YOUR ARTICLE

%%%%%%%%%%%%%%%%%%%%%%% References %%%%%%%%%%%%%%%%%%%%%%%%%


%\bibliographystyle{osajnl}   %>>>> makes bibtex use osajnl.bst
%\bibliography{references}   %>>>> bibliography data in references.bib

% \begin{thebibliography}{99}

% \bibitem{gallo99} K. Gallo and G. Assanto, ``All-optical diode based on second-harmonic generation in an asymmetric waveguide,'' \josab {\bf 16,} 267--269 (1999).

% \end{thebibliography}


%\newpage
%%%%%%%%%%%%%%%%%%%%%%%%%%%%%%%%%%%%%%%%
%------------section--------------------
%%%%%%%%%%%%%%%%%%%%%%%%%%%%%%%%%%%%%%%%
\newpage
        \section{Fundamentals of the LQG control approach }\label{sec:reviewLQG}

        The linear-quadratic (LQ) and linear-quadratic-Gaussian (LQG)
        approaches are two standard design tools in optimal control
        theory~\cite{barshalom74,andersonmoore_optimalcontrolLQG05}
        that stand on a state-space formalism. 

        This formalism
        consists in mathematically modelling a
        physical system as a set of inputs, outputs and state variables
        related by first order difference equations (discrete case). States are
        considered as column vectors and the algebraic and difference
        equations are written in matrix form. With respect to the classical
        single-input single-output (SISO) systems for which frequency
        analysis is often preferred (in the form of
        transfer-functions) using state-spaces is a very
        convenient manner to represent multiple-input multiple-output
        (MIMO) systems with real-time
        computational delays. In an AO context, phase (and its temporal
        dynamics) along with the WFS measurements and DM commands
        can be explicitly included in the model.

        % Regulator synthesis in AO was hitherto based upon modal
        % decoupling (see for example~\cite{gendron94}). In so doing,
        % one employs as many scalar regulators as
        % controlled modes. Frequency analysis is therefore very
        % handy. The commonly-used integrators with
        % gain (proportional-integral action) can be straightforwardly
        % embedded in
        % a more general state-space approach, which brings to light
        % their internally-hidden unstable phase
        % temporal model~\cite{kulcsar06}. Though simple to synthesize,
        % attainable performance is compromised. 

        The initial LQG setup relies on three essential requirements:
        $1)$ the system is linear, $2)$ the criterion is quadratic and
        $3)$ the disturbances are zero-mean, white and Gaussian-distributed.
        Consider the following general discrete-time linear system of
        the order $n$, $\forall k \geq 0$,   
        \begin{align}\label{eq:complete_discrete_state_space_system}
          \left\{
            \begin{array}{cl}
              \xvec_{k+1}  &  =\mathcal{A}_{\mathrm{d}}\xvec_{k}+\mathcal{B}_{\mathrm{d}} \uvec_{k}+\Vd v_{k}\\
              % \overline{\phivec}_{k+1}^{\tur} & = \mathcal{C}_{\mathrm{d}}^{\tur}\xvec_k \\
              \svec_{k} & = \mathcal{C}_{\mathrm{d}} \xvec_{k} + \mathcal{D}_{\mathrm{d}} \uvec_{k} + w_k %\\
              %z_{k}^{\crit}  &  =\mathcal{C}_{\mathrm{d}}^{\crit} \xvec_{k} 
            \end{array}\right. ,
        \end{align}
        The variables $v$ and $w$ are assumed to be 
        Gaussian white noises, \emph{i.e.} sequences of independent and identically 
        distributed vector-valued Gaussian variables, with distributions $v_k 
        \sim \mathcal{N}(0,\Sigma_v)$ and $w_k \sim
\mathcal{N}(0,\Sigma_w)$. 

        In Eq.~(\ref{eq:complete_discrete_state_space_system}) the notion of
        state gathers both the disturbance and
        mirror states.
        The general system of
        Eq.~(\ref{eq:complete_discrete_state_space_system}) defines
         \begin{enumerate}
          \item a state evolution equation, where the state temporal
            dynamics is characterised by a state transition matrix
            $\mathcal{A}_{\mathrm{d}}$ and an input vector $\mathcal{B}_{\mathrm{d}}$ that
            brings about the effect of the control decisions $\uvec_k$ at
            instants $t=kT_s$ and state noise
            $v_k$ a zero-mean white noise with covariance 
            matrix $\Sigma_{v}$. This latter component will
            be used to account for the
            random behaviour of disturbance;
          \item a measurement equation indicating which information is
            available at $t=kT_s$ for computing the control $\uvec_k$. The
            measurement $z_k$ is assumed to be the sum of a linear
            function of the internal state $\xvec_k$  corrupted with additive zero-mean Gaussian white
            noise $w_k\sim
            \mathcal{N}(0,\Sigma_{w})$ independent of $v_k$ (and
            hence of $\xvec_k$).
        \end{enumerate}


    % ------------section--------------------
\subsection{Optimality criteria }

   
	% ------------subsection-----------------
	%\subsection{Minimum pupil-integrated mean-square error}
        In what follows, the minimisation of the
        pupil-integrated mean-square residual phase after AO
        correction is considered. Minimising the variance of  $\phivec^{\res}$ % $J^{\mathrm{c}}$ in Eq~(\ref{eq:Cont_criterion})   
        results in the maximisation of the Strehl-ratio
        (SR)~\cite{herrman92} leading to the continuous-time criterion   
	\begin{equation}\label{eq:Cont_criterion}
          J^{\mathrm{c}}\left(  u\right)
          \triangleq\lim_{\tau\rightarrow+\infty}\frac{1}
          {\tau}\int_{0}^{\tau}\left\Vert \phivec^{\res}\left(
              t\right)
          \right\Vert^{2}dt=\lim_{\tau\rightarrow+\infty}\frac{1}{\tau}\int_{0}^{\tau}\left\Vert\phivec^{\tur}\left( 
              t\right) -\phivec^{\cor}\left(  t\right)\right\Vert
          ^{2}dt,
	\end{equation}
	where the residual phase is the difference between the
        turbulent and correction phases, $\phivec^{\res} = \phivec^{\tur} -
        \phivec^{\cor}$. % Refer to Fig. \ref{fig:STANDARD_diagram} for further details
        % on the feedback control system.  


        % From a control point-of-view, the objective of
        % Eq.~(\ref{eq:Cont_criterion}) defines a disturbance rejection
        % problem, since the control $u$ cannot act on the disturbance
        % phase $\phivec^{\tur}$. 

        %(On simple physical grounds this can be straightforwardly understood.)

        
        Define $J^\mathrm{c}(u)_k$ as the average value of the residual phase
        variance between sampling instants $t=kT_s$ and $t=(k+1)T_s$, 
	\begin{equation}\label{eq:Cont_criterion_stepbystep}
          J^\mathrm{c}\left(u\right)_k  \triangleq  \frac{1}{{T_s}} \int_{k{T_s}}^{(k+1){T_s}} \left\|\phivec^{\res}(t)\right\|^2 dt = \frac{1}{{T_s}} \int_{k{T_s}}^{(k+1){T_s}} \left\|\phivec^{\tur}(t) - \phivec^{\cor}(t)\right\|^2 dt.
	\end{equation}
        With this notation, the continuous-time criterion of
        Eq.(\ref{eq:Cont_criterion}) can be rewritten as an
        infinite-horizon average over
        all temporal frames of duration ${T_s}$ 
	\begin{equation}\label{eq:Jc_average_Jck}
          J^\mathrm{c}(u) = \lim_{M\rightarrow+\infty}\frac{1}{M}\sum_{k=0}^M J^\mathrm{c}(u)_k.
	\end{equation} 
        Note also that by construction $J^\mathrm{c}(u)_k$ depends only on
        control decisions made up to time $t=kT_s$, in other words on
        the discrete sequence $\uvec_0,\ldots, \uvec_k$.
        
        % Under the standard assumption
        % that $\phivec^{\tur}$ is a wide-sense stationary stochastic process, and when a
        % stabilising linear time-invariant controller is used to generate the
        % control $u$, it is immediately shown that the stochastic process
        % ${J^\mathrm{c}(u)}_{k \in \mathbb{N}}$ converges towards a stationary and ergodic process.
        Under standard assumptions, the criterion $J^\mathrm{c}(u)$ in Eq.(\ref{eq:Cont_criterion}) is 
        almost surely (a.s.) equal to the mathematical expectation of
        $J^\mathrm{c}(u)_k$ 
        \begin{equation}
          J^\mathrm{c}(u) \overset{a.s.}{=} \text{E}\left( J^\mathrm{c}(u)_k \right)
        \end{equation}

% The LQG solution for the MV AO control problem that is extended here
% has been  previously established in the
% infinitely fast DM case~\cite{leroux04, petit04, kulcsar06, kulcsar08}. For introductory
% purposes, it is recalled below. The CSI case is considered first. 

   % ------------section--------------------
\subsection{Step-by-step solution }

   
        Within a LQ/LQG framework, the problem at hand can be phrased
        as the determination of the control 
        decisions $\uvec_k~\forall k>0$ that minimise a quadratic
        criterion in the form
        \begin{equation}\label{eq:Jd_general_LQG}
          J^\mathrm{d}\left(  u\right)
          =\lim_{M\rightarrow+\infty}\frac{1}{M}\sum_{k=0}^{M-1}\left(
             \xvec_{k}^{\T}Q
            \xvec_{k}+\uvec_{k}^{\T}R \uvec_{k}+2\xvec_{k}^{\T}S \uvec_{k}\right) ,
        \end{equation}
        where $Q$, $S$ and$ R$ are weighting matrices
        to be obtained from the development of the AO MV criterion. To define
        a well-posed MV problem, it is necessary that
        Eq.~(\ref{eq:Jd_general_LQG}) corresponds a a proper
        quadratic form, \emph{i.e.}
          \begin{equation}\label{lq_wheigh_matrices}
            P\triangleq\left(
              \begin{array}
                [c]{cc}%
                Q &  S\\
                 S^{\T} &  R
              \end{array}
            \right)  \geq 0.
          \end{equation} 
        This positiveness condition is in turn equivalent to $Q=Q^{\text
          T}$, $Q-SR^{-1}S^{\T}\geq0$ $S$ and $R=R^{\T}>0$.

        Suppose that the  state $\xvec_k$  $\forall
        k>0$ is perfectly known.  
        The solution to this \emph{complete state information} (CSI)
        problem can be expressed in state feedback form as
        \begin{equation}\label{eq:u_k_optimal}
          \uvec_k = -\K\xvec_k ,
        \end{equation}
        where $\K$ is found from the solution of
        a control Riccati equation that involves the matrices $Q$, $R$
        and $S$ in Eq.~(\ref{eq:Jd_general_LQG}).

        In the general and more realistic \emph{incomplete state
          information} (ISI) case the \emph{separation
          principle}~\cite{barshalom74,andersonmoore_optimalcontrolLQG05}
        applies: the optimal control has the same form as 
        Eq.~(\ref{eq:u_k_optimal}), where the state is replaced by its
        conditional expectation, denoted $\widehat{\xvec}$. Such state is
        estimated from the noisy and possibly delayed measurements%  -
        % see the temporal arrangement of events in
        % Fig.~\ref{fig:chronogramme}
        . 
        
        In other words, the optimal control decisions are retrieved from   
        \begin{equation}\label{eq:u_k_optimal_est}
          \uvec_k = -\K \widehat{\xvec}_{k|k} ,
        \end{equation}
        using this estimate as if it were the true state in the
        state-feedback control. The subscript in the form $(k|k-1)$ means the
        conditional expectation of $\xvec_{k}$ with respect
        to statistical priors and measurements $\mathcal{Y}_k \triangleq\{y_{0},\ldots
        ,y_{k}\}$.

        Provided one has a linear model describing the state's
        dynamics, $\widehat{\xvec}_{k|k-1}$ is obtained recursively from the
        output of a Kalman
        filter
        with gain $\mathcal{L}^{\text{opt}}$~\cite{andersonmoore_optimalcontrolLQG05}.
        The state estimate is obtained from the recursive equation
        \begin{equation}\label{eq:kalman_state_prediction}
          \widehat{\xvec}_{k|k-1} = 
          \mathcal{A}_\mathrm{d}\widehat{\xvec}_{k-1|k-2}+\mathcal{B}_\mathrm{d}
          \uvec_{k-1}+\Lkal\left(\svec_{k-1}-\widehat{\svec}_{k-1|k-2}\right) ,
        \end{equation}
        where the gain
\begin{equation}\label{eq:KalmanGain}
          \Lkal	= 
          \mathcal{A}_\mathrm{d} \CovMat\mathcal{C}_\mathrm{d}^\T \left(\mathcal{C}_\mathrm{d}
            \CovMat\mathcal{C}_\mathrm{d}^\T + \CovMat_w\right)^{-1} 
        \end{equation}
        is obtained from
        \begin{equation}\label{eq:ARE}
          \CovMat = \mathcal{A}_\mathrm{d} \CovMat\mathcal{A}_\mathrm{d}^\T + \CovMat_v - \mathcal{A}_\mathrm{d} \CovMat\mathcal{C}_\mathrm{d}^\T\left(\mathcal{C}_\mathrm{d} \CovMat\mathcal{C}_\mathrm{d}^\T + \CovMat_w\right)^{-1}\mathcal{C}_\mathrm{d}\CovMat \mathcal{A}_\mathrm{d}^\T
        \end{equation}
        In summary, the full LQG
        solution is found in two steps: first solve a deterministic control
        problem establishing the optimal regulator in the CSI case, and then
        solve a MV estimation problem to find $\widehat{\xvec}_{k|k-1}$ in the
        ISI case % - Fig.~\ref{fig:optimal_regulator}
        .
        
        % To evaluate $J^\mathrm{c}(u)$ in (\ref{eq:Cont_criterion_stepbystep})
        % three actions are required; they are to be followed in
        % the subsequent sections: 
        % \begin{enumerate}
        % \item further develop the quadratic criterion to retrieve 
        %   the weighting matrices $Q$, $R$ and $S$ by comparing
        %   Eq.~(\ref{eq:Cont_criterion_stepbystep}) to 
        %   Eq.~(\ref{eq:Jd_general_LQG});
        % \item with these determine the state feedback and observer gains
        %  $\K$ and $\mathcal{L}^{\text{opt}}$ and are
        %   found from the solution of a control and estimation Riccati
        %   equations respectively, upon
        %   full construction of the state-space system of
        %   Eq.~(\ref{eq:complete_discrete_state_space_system})~(in
        %   sections~\ref{sec:complete_info_state_model}-\ref{sec:incomplete_info_state_vector});
        % \item finally,
        % the residual phase variance can be analytically evaluated~(in
        % section~\ref{sec:evaluation_criterion}). 
        % \end{enumerate}
        
        % Next section
        % establishes the LQG control solution for AO systems with
        % infinitely-fast DM as a first step towards the more complex
        % case with slower DM.


%%%%%%%%%%%%%%%%%%%%%%%%%%%%%%%%%%%%%%%%
%------------section--------------------
%%%%%%%%%%%%%%%%%%%%%%%%%%%%%%%%%%%%%%%%
\section{LQG controller for AO systems with infinitely fast
  mirrors}\label{sec:infinetely_fast_dm}
        
        \subsection{Solution in the \emph{complete-state information} case}
        When the DM's transient response is negligible compared with the sampling
        period ${T_s}$ so that its dynamics can be reduced to its DC gain
        matrix $\N \in \Re^{n\times m}$
        (the so-called ``DM influence matrix'') then 
        \begin{equation}\label{eq:phicorr_from_u}
          \phivec ^{\cor}\left( t\right)
          =\N \uvec_{k},~ ~ ~  \forall~~ k{T_s}\leq t<\left( k+1\right)
          {T_s} .
        \end{equation}
        As shown in \cite{ kulcsar06},  the globally optimal MV
        control - that is, the discrete control 
        minimising the continuous-time criterion $J^\mathrm{c}(u)$ - is obtained by
        minimising the degenerate (because independent from the
        phase temporal evolution model, just the inter-sample phase
        mean value is used) equivalent discrete-time MV
        criterion\cite{leroux04, petit04}, denoted by
        $J_{\text{ifm}}^{\mathrm{d}}$, with subscript standing for \emph{infinitely
        fast mirror}, 
        \begin{align}
          J_\text{ifm}^{\mathrm{d}}\left(u\right)   &
          =\lim_{M\rightarrow+\infty}\frac{1}{M}\sum_{k=0}^{M-1}\Vert\overline{\phivec}_{k+1}^{\tur}
          - \N \uvec_k \Vert^2 ,
          \label{eq:Disc_criterion}
	\end{align}
        where $J^\mathrm{c}(u)$ relates to $J^\mathrm{d}_\text{ifm}(u)$ by
        \begin{equation}\label{eq:insurmontableerror}
        J^\mathrm{c}(u) = J^\mathrm{d}_\text{ifm}(u) + \delta J({T_s}) ,
	\end{equation}
        with $\delta J({T_s})$ the insurmountable
        error due to the use of averaged variables
        instead of continuous ones~\cite{petit06}
        \begin{equation}\label{eq:unsurmountable_error}
          \delta J\left(   T_s\right)  =\lim_{M\rightarrow+\infty}\frac{1}{M}%
          \sum_{k=0}^{M-1}\left(  \frac{1}{T_s}\int_{k T_s}^{\left(
                k+1\right)  \ T_s}\left\Vert \phivec^{\tur}\left( 
                t\right) - \overline{\phivec}_{k+1}^{\tur}
            \right\Vert ^{2}dt\right).
        \end{equation}
  Symbol $\overline{\cdot}$ represents
time average, \textit{i.e.}
\begin{equation}\label{eq:av_meas}
 \overline{\boldsymbol\phivec}_k \triangleq \int_{(k-1)T_s}^{kT_s} \boldsymbol\phivec(\tau)\dint \tau
\end{equation}


For the commonly used frame-rates of several hundred Hertz,
        $\delta J({T_s})$ is small and normally accounted for in the total
        error budget. % Figure \ref{fig:unsurmontable_error} shows t
        The absolute error $ \delta J\left(   T_s\right)$ can be computed from
\begin{equation}\label{eq:unsurmountable_error}
          \delta J\left(   T_s\right)  = \int_{-\infty}^\infty
          \left(1 - |sinc(\kappa T_s)|^2\right) PSD_{TT}(\kappa) \dint f.
        \end{equation}
for three combinations of atmospheric and wind conditions. At a
nominal sampling frequency of 1500Hz, in bad conditions the
insurmountable error is below 0.01 mas$^2$. This is below 1\% the
achieved performance (if taken to be 4 mas rms in general) and it is
the improvement a fractional-delay controller is to  bring about.
% \begin{figure}[htpb]
% 	\begin{center}
% 	% use packages: array            
%         %  \begin{tabular}{cc}
%             \includegraphics[width=1.0\textwidth]{insurmontable_error.eps}
%          % \end{tabular}
% 	\end{center}
% 	\caption[]
% 	{\label{fig:insurmontable_error}
%           Insurmountable error as a function of the AO sampling frequency.}
% \end{figure}



        For the sake of simplicity, it shall be assumed that the DM
        influence matrix $\N$ 
        is full column rank, in other words that $\N^{\T}\N$ is invertible.
        Since $J_\text{ifm}^\mathrm{d}(u)$  relates to
        $J^\mathrm{c}(u)$ by a term independent of the control decision $\uvec$ in
        Eq.~(\ref{eq:insurmontableerror}), 
        it follows that $\arg\min_{\mathcal{U}}
        J^{\mathrm{c}}\left(  u\right)  =\arg\min_{\mathcal{U}}
        J_\text{ifm}^{\mathrm{d}}\left(  u\right)$, with $\mathcal{U}$ the
        set of admissible controls, which establishes the
        equivalence of both criteria.  
        
        The optimal control decisions $\uvec_k$ in the CSI case are hence
        given by 
	\begin{equation}\label{eq:CSIcontrol_from_simple_projection}
          \uvec_k = \Theta \overline\phivec_{k+1} = \left(\N^{\T}\N\right)^{-1}\N^{\T}\overline\phivec_{k+1}^\tur,
	\end{equation}
	with $\Theta \in \Re^{m\times n}$ the orthogonal
        projector of phase onto the DM's space.
        
        Two interesting features of the CSI control are worth noting:
        1) it does not depend on any particular assumption on the
        turbulent phase and
        2) it effectively redefines \emph{complete state information}
        as the advance knowledge not 
        of the whole turbulent phase trajectory over the next sampling
        interval, but 
        rather of its average value $\overline\phivec_{k+1}^\tur$. The latter is specially
        interesting for the ISI case covered next.

        \subsection{Solution in the \emph{incomplete state
            information} case}

        In the \emph{incomplete state
            information} (ISI) case the state is estimated from WFS measurements. These measurements can also be expressed as functions of
        the average phase
        $\overline{\phivec }^{\tur}$ since light flux emanating from
        a guide star is commonly integrated during the detector
        exposure time.
        Assuming a linear response, their output can be
        modeled to a first approximation as

        \begin{equation}\label{eq:pseudomeasurement_dyn}
          \svec_{k} = \D \overline{\phivec }_{k}^{\tur}-\D\N \uvec_{k-1}+w_{k}
        \end{equation}
        where $w_k$ is a zero-mean, Gaussian white noise $w_k \sim
        \mathcal{N}(0, \CovMat_{w})$. The assumptions on the WFS
        read-out and processing delays are not limiting - the
        approach presented here could easily be adapted to cope with a
        different temporal diagram. In
        Eq.~(\ref{eq:pseudomeasurement_dyn}), the linear operator 
        $\D \in \Re^{s \times n}$ is a device-specific phase-to-WFS 
        influence matrix and $N$ is defined in
        Eq.~(\ref{eq:phicorr_from_u}). 

        As a consequence, \emph{any} stochastic model of $\overline{\phivec }^{\tur}$ will enable to construct an
        ``exhaustive control-oriented'' model for the AO loop - \emph{i.e.}, one from
        which the globally optimal MV control can be derived using standard LQG
        procedures.\ 
        
        Assuming that $\overline{\phivec }^{\tur}$ is a wide sense stationary
        random process, the \textit{separation
          principle}
        applies; the MV control then becomes 
        \begin{equation}\label{eq10}
          \uvec_{k}=\Theta\widehat{\overline{%
              \phivec}}_{k+1\mid \mathcal{Y}_k}^{\tur},
        \end{equation}%
        where $\widehat{\overline{\phivec }}_{k+1\mid \mathcal{Y}_k}^{\tur}$
        denotes the conditional expectation of $\overline{\phivec }_{k+1}$
        with respect to the sequence of all measurements available at
        $t=kT_s$, namely  $\mathcal{Y}_k = \{y_{0},\ldots ,y_{k}\}$. 

      Any state-space model meant
          to describe these measurements should accommodate  $\overline\phivec_{k+1}^\tur$ to
          compute the optimal control decisions in 
          Eq.~(\ref{eq:CSIcontrol_from_simple_projection}) and further 
          $(\overline\phivec_{k}^\tur,\uvec_{k-1})$ in
          Eq.~(\ref{eq:pseudomeasurement_dyn}). Consider
          the average phase
          temporal evolution if given by a
          first order
          auto-regressive model 
          \begin{equation}
            \overline{\phivec}^\tur_{k+1} = \mathcal{A}_{\tur} \overline{\phivec}^\tur_{k} + v_k.
          \end{equation}
          Though these simplistic models cannot thoroughly represent Taylor's frozen
          flow hypothesis, they are accurate enough to grasp the spatial
          (inter-modal) statistics and the short term correlations.
          
          All gathering in a complete state, the full
          discrete-time state-space becomes 
          \begin{equation}\label{ss_system_nodynamics}
            \left(
              \begin{array}
                [c]{c}%
                \overline\phivec_{k+2}^{\tur} \\ \overline\phivec_{k+1}^{\tur} \\ \uvec_{k}
              \end{array}	
            \right)
            = 
            \left(
              \begin{array}
                [c]{ccccc}%
                \mathcal{A}_{\tur} & 0 & 0 &\\
            \I & 0 & 0 \\
            0 & 0 & 0 \\
          \end{array}	
        \right)
        \left(
          \begin{array}
            [c]{c}%
            \overline\phivec_{k+1}^{\tur} \\ \overline\phivec_{k}^{\tur} \\ \uvec_{k-1}
          \end{array}	
        \right) 
        +
        \left(
          \begin{array}
            [c]{c}%
            0 \\ 0 \\ \I 
          \end{array}	
        \right)
        \uvec_k
        +
        \left(
          \begin{array}
            [c]{c}%
            \I \\ 0 \\ 0 
          \end{array}	
        \right)
        v_k ,
      \end{equation} 
      with 
      \begin{equation}\label{eq:pseudomeasurement_nodyn}
        \svec_k = \D
        \left(
          \begin{array}
            [c]{ccccc}%
             0 & \I &  -\N
          \end{array}	
        \right) 
        \xvec_k + w_k .
      \end{equation}
      The control decisions are found from 
      \begin{equation}\label{eq:optimalcontrols_nodyn}
        \uvec_k = \Theta
        \left(
          \begin{array}
            [c]{ccccc}%
            \I  & 0 & 0
          \end{array}	
        \right) 
        \widehat{\xvec}_{k|k} ,
      \end{equation}
      where the estimation version of the controller uses information
      up to $\svec_{k}$. This ensures that the loop delays are correctly taken
      into account in the complete model of Eqs.~(\ref{ss_system_nodynamics}-\ref{eq:optimalcontrols_nodyn}).
      
      Expanding Eq.~(\ref{eq:Disc_criterion}) and comparing it to
      Eq.~(\ref{eq:Jd_general_LQG}) the LQ weighting matrices are
      readily
      \begin{equation}
        Q = T^{\T}T \geq 0,  ~~~ S = -\N T^{\T},    ~~~ R = \N^{\T}\N
        > 0 ,
      \end{equation}
      and $T = \left(\begin{array}
          [c]{ccccc}%
          \I & 0 &  0
        \end{array}\right)$. This yields an equivalent discrete-time
      criterion of the form
      \begin{align}
        J_\text{ifm}^{\mathrm{d}}\left(u\right) & =\lim_{M\rightarrow+\infty}\frac{1}{M}\sum_{k=0}^{M-1}\left(
          \left( 
            \begin{array}
              [c]{c}%
              \overline{\phivec}_{k+1}^{\tur}\\
              \uvec_{k}%
            \end{array}
          \right)  ^{\T}\left(
            \begin{array}
              [c]{cc}%
              \I & -\N\\
              -\N^{\T} & \N^{\T} \N
            \end{array}
          \right)  
          \left(
            \begin{array}
              [c]{c}%
              \overline{\phivec}_{k+1}^{\tur}\\
              \uvec_{k}%
            \end{array}
          \right)  \right) .
        \label{eq:Disc_criterion_ISI}
      \end{align}
      In this case, the optimal control gain,
      as shown above, can be computed directly as an orthogonal projection.
      The optimal observer gain (Kalman gain), on the other hand, is
      computed from an estimation Riccati equation. % This observer gain is to

        \subsection{Extension to vibration suppression}


Let the standard second-order differential equation
\begin{equation}\label{eq:2nd_order_diff_eq}
  \ddot\theta + 2\xi \omega_0^2 \dot \theta + \omega_0^2 \theta = \nu 
\end{equation}
where  $\theta$ is the vibration
signal, $\xi$ is a damping factor (defines the width and overshoot of the vibration
peak) and $\omega_0$ is the natural oscillatory frequency in
rad/s. $\nu$ is a continuous excitation signal, chosen here to be a zero-mean
Gaussian white noise.

The solution to the ordinary differential-equation is an oscillatory response in the form
\begin{equation}
  \theta = \frac{e^{-\xi\omega_0 t}}{\sqrt{1-\xi^2}}sin\left(2\pi\sqrt{1-\xi^2}\omega_0 t + \Delta \varphi\right) 
\end{equation}
where $\Delta \varphi$ is a phase shift found from the initial
conditions set to Eq. \eqref{eq:2nd_order_diff_eq}.

\cite{petit08} and \cite{meimon10} used a discretised version of the
vibration given by \eqref{eq:2nd_order_diff_eq} that led to the $\mathcal Z-$domain model
\begin{equation}\label{eq:TF_disc_vib}
  H(z) = \frac{\varepsilon_k}{1-2e^{-\xi \omega_0^2T_s} cos\left(\omega_0 T_s \sqrt{1-\xi^2}\right)z^{-1} +e^{-2\xi \omega_0^2T}z^{-2} }
\end{equation}
where $T_s$ is the sampling interval. 

The correspondence with a second-order auto-regressive formulation is
straightforward (using standard $\mathcal Z-$transform properties, \cite{oppenheim97})
\begin{subequations}
 \begin{align}\label{eq:ar2_vibration}
 \theta_{k+1} & = \alpha_1  \theta_{k} + \alpha_2  \theta_{k-1} + \varepsilon_k \\
\alpha_1 & = 2e^{-\xi \omega_0^2T_s} cos\left(\omega_0 T \sqrt{1-\xi^2}\right) \\
\alpha_2 & = e^{-2\xi \omega_0^2T_s}
\end{align}
\end{subequations}
where $\epsilon_k$ is the process excitation noise, considered zero-mean,
Gaussian-distributed and temporally white.

% Equation~\eqref{eq:TF_disc_vib} admits the equivalent
% discrete-time state-space representation (time-dependence omitted):\begin{align} \label{eq:ss_cont_vibration}
%   \left\{
%     \begin{array}{cl}
%        x^{\vib}_{k+1}  &  ={A}_v \xvec_k^{\vib}+{B}_v \uvec _{k}^\vib+\Gamma_\vib  \varepsilon _{k}\\
%      \theta _{k} & = {C}_\vib \xvec_{k}^{\vib 
%     \end{array}\right.
% \end{align}
% with
% \begin{subequations}
% \begin{align}\label{ss_ar2_vibration_w_delay}
%   \underbrace{\left(
%     \begin{array}
%       [c]{c}%
%       \theta_{k+2} \\ \theta_{k+1} 
%     \end{array}	
%   \right)}_{\xvec_{k+1}^\vib}
%    & = 
%   \underbrace{\left(
%     \begin{array}
%       [c]{ccccc}%
%       \alpha_1& \alpha_2  \\
%       1 & 0 
%     \end{array}	
%   \right)}_{\mathcal{A}_\vib}
%   \underbrace{\left(
%     \begin{array}
%       [c]{c}%
%        \theta_{k+1} \\ \theta_{k} 
%     \end{array}	
%   \right)}_{\xvec_{k}^\vib} 
%   +
%  \underbrace{ \left(
%     \begin{array}
%       [c]{c}%
%       1 \\ 0 
%     \end{array}	
%   \right)}_{\Gamma_\vib}
%   \varepsilon_k
%  \\
%   \theta_k & = 
%   \underbrace{\left(
%     \begin{array}
%       [c]{ccccc}%
%       0  & 1 
%     \end{array}	
%   \right)}_{\mathcal{C}_\vib} 
%    \xvec_{k}^\vib
% + w_k
% \end{align}
% \end{subequations}

% The current GPI representation is as follows \cite{poyneer10}
% \begin{subequations}
% \begin{align}\label{lisa_vib_model}
%   \underbrace{\left(
%     \begin{array}
%       [c]{c}%
%       \theta^r_{k+1} \\ \theta^i_{k+1} 
%     \end{array}	
%   \right)}_{\xvec_{k+1}^v}
%    & = 
%   \underbrace{\left(
%     \begin{array}
%       [c]{ccccc}%
%       \alpha_r& -\alpha_i  \\
%       \alpha_i & \alpha_r 
%     \end{array}	
%   \right)}_{\mathcal{A}_v}
%   \underbrace{\left(
%     \begin{array}
%       [c]{c}%
%        \theta^r_{k} \\ \theta^i_{k} 
%     \end{array}	
%   \right)}_{\xvec_{k}^v} 
%   +
%  \underbrace{ \left(
%     \begin{array}
%       [c]{cc}%
%       1 &0\\ 0&1 
%     \end{array}	
%   \right)}_{\Gamma_v}
%   \left(
%     \begin{array}
%       [c]{c}%
%       \varepsilon_k^r \\ \varepsilon_k^i 
%     \end{array}	
%   \right)
%  \\
%   \theta_k & = 
%   \underbrace{\left(
%     \begin{array}
%       [c]{ccccc}%
%       1  & 0 
%     \end{array}	
%   \right)}_{\mathcal{C}_v} 
%    \xvec_{k}^v
% \end{align}
% \end{subequations}

%%%%%%%%%%%%%%%%%%%%%%%%%%%%%%%%%%%%%%%% 
% ------------section--------------------
%%%%%%%%%%%%%%%%%%%%%%%%%%%%%%%%%%%%%%%% 
% \subsubsection{Equivalence vibration models}\label{sec:equiv-vibration-models}

% Lisa's model is equivalent to a second order with variable damping.

% Therefore, it needs be identified and not pre-established.


% Since the characteristic polynomial of these two models has two
% complex-conjugate roots, they are equivalent, with (for $0<\xi\leq1$)
% \begin{subequations}
% \begin{align}
% \alpha_1 & = 2\alpha^r\\
% \alpha_2 & = \left(\alpha^r\right)^2 + \left(\alpha^i\right)^2
% \end{align}
% \end{subequations}
% {\color{red} This is not exactly it because though the characteristic
%   polynomial is the same, the zeros are not the same. There is indeed
%   a slight difference, to assess later...}





%%%%%%%%%%%%%%%%%%%%%%%%%%%%%%%%%%%%%%%% 
% ------------section--------------------
%%%%%%%%%%%%%%%%%%%%%%%%%%%%%%%%%%%%%%%% 
\section{NGS modes in laser-tomography AO: LTAO and MCAO}

%{}

The NGS modes in laser-tomography AO are defined as the null modes
of the high-order LGS measurement space, i.e., modes that produce
average slope $\neq$ 0, but that due to the LGS tilt indetermination,
cannot be measured by the latter. In other words, the null space can
be thought as the combination of all the modes that have non-null
projection onto the angle-of-arrival ( = not just Zernike tip and tilt
but also higher order Zernike modes).


%------------section--------------------
\subsection{Measurement model with time-averaged variables }\label{sec:measurement-model}

Assume the following measurement model 
\begin{subequations}\label{eq:meas_quadratic_modes}
\begin{align}
\boldsymbol{s}_k & = \int_{(k-1)T_s}^{kT_s} \D\left(  \boldsymbol\psi (\tau)^\tur - 
   %P_\textsf{\tiny{N}}
  \boldsymbol\psi^{\,\,\cor} (\tau) \right) \dint \tau + \boldsymbol\eta_k \\
& = \D \left(  \overline{\boldsymbol\psi^\tur _{k}} - 
  \overline{\boldsymbol\psi^\cor_{k}} \right) + \boldsymbol\eta_k \\
& = \D   \overline{\boldsymbol\psi^\res_k} + \boldsymbol\eta_k
\end{align}
\end{subequations}
where  $\boldsymbol{s}_k \in \Re^{12\times1}$ are the $T_s$-averaged
       slopes over each OIWFS sub-aperture and over the integration
       time $T_s$, $\D \in \mathbb{R}^{12\times 9}$ is the wave-front-to-measurements
matrix, 'dm' and 'res' stand respectively for
correction and residual phase;  $\boldsymbol{\eta}_k$ is a zero-mean
Gaussian-distributed spectrally white noise vector with known covariance matrix
$\CovMat_\eta$ --
$\boldsymbol\eta\sim \mathcal{N}(0,\CovMat_\eta)$.

The modal matrix $\D$ translates modal coefficients of TT, TT and TTFA
modes into average slopes over the illuminated sub-region of
each sub-aperture
\begin{align}\label{eq:Gamma}
          \D & \triangleq \left( 
            \begin{array}{c}
              \D_\textsf{TT1} \\\hdashline
              \D_\textsf{TT2}\\\hdashline
              \D_\textsf{TTFA}
            \end{array}%
          \right) \nonumber \\ & = 
\left( 
            \begin{array}{cc:cc:ccccc}
              \gamma_\textsf{T}  & \0 & \0 & \0 & \0 & 0 & \0 & 0 & \0\\ 
              \0 & \gamma_\textsf{T}   & \0 & \0 & \0 & \0 & 0 & 0 & \0\\ \hdashline
              0 & \0 & \gamma_\textsf{T}  & \0 & \0 & 0 & \0 & 0 & \0\\ 
              \0 & 0  & \0 & \gamma_\textsf{T}  & \0 & \0 & 0 & 0 & \0\\ \hdashline
              \0 & 0  & \0 & 0 & \gamma_\textsf{T}  & \0 & \gamma_\textsf{F} & \gamma_\textsf{A} & \gamma_\textsf{A}\\
              \0 & 0  & \0 & 0 &  0 & \gamma_\textsf{T} & \gamma_\textsf{F} & \gamma_\textsf{A} & -\gamma_\textsf{A}\\  
               \0 & 0  & \0 & 0 & \gamma_\textsf{T} & 0 & -\gamma_\textsf{F} & \gamma_\textsf{A} & -\gamma_\textsf{A}\\
               \0 & 0  & \0 & 0 & 0 & \gamma_\textsf{T} & \gamma_\textsf{F} & -\gamma_\textsf{A} &-\gamma_\textsf{A}\\ 
               \0 & 0  & \0 & 0 & \gamma_\textsf{T} & 0 & -\gamma_\textsf{F} & -\gamma_\textsf{A} & -\gamma_\textsf{A}\\
               \0 & 0  & \0 & 0 & 0 & \gamma_\textsf{T} & -\gamma_\textsf{F} & -\gamma_\textsf{A} & \gamma_\textsf{A}\\ 
               \0 & 0  & \0 & 0 & \gamma_\textsf{T} & 0 & \gamma_\textsf{F} & -\gamma_\textsf{A} & \gamma_\textsf{A}\\
               \0 & 0  & \0 & 0 & 0 & \gamma_\textsf{T} & -\gamma_\textsf{F}  & \gamma_\textsf{A} & \gamma_\textsf{A}  
            \end{array}%
          \right),% \left( 
          %   \begin{array}{cc:cc:ccc}
          %     2 & \0 & \0 & \0 & \0 & 0 & \0 \\ 
          %     \0 & 2  & \0 & \0 & \0 & \0 & 0 \\ \hdashline
          %     0 & \0 & 2 & \0 & \0 & 0 & \0\\ 
          %     \0 & 0  & \0 & 2 & \0 & \0 & 0 \\ \hdashline
          %     \0 & 0  & \0 & 0 & \gamma_\textsf{T}  & \0 & -\gamma_\textsf{F} \\
          %     \0 & 0  & \0 & 0 &  0 & \gamma_\textsf{T} & \gamma_\textsf{F} \\  
          %      \0 & 0  & \0 & 0 & \gamma_\textsf{T} & 0 & \gamma_\textsf{F} \\
          %      \0 & 0  & \0 & 0 & 0 & \gamma_\textsf{T} & \gamma_\textsf{F} \\ 
          %      \0 & 0  & \0 & 0 & \gamma_\textsf{T} & 0 & -\gamma_\textsf{F} \\
          %      \0 & 0  & \0 & 0 & 0 & \gamma_\textsf{T} & -\gamma_\textsf{F} \\ 
          %      \0 & 0  & \0 & 0 & \gamma_\textsf{T} & 0 & \gamma_\textsf{F} \\
          %      \0 & 0  & \0 & 0 & 0 & \gamma_\textsf{T} & -\gamma_\textsf{F}    
          %   \end{array}%
          % \right)
        \end{align}
% where the $2's$ represent TT measurements directly sensed by both the
% TT-WFSs. Given the first two equations of
% Eq. \eqref{eq:TT_PS_definition}, the average gradient over the
% aperture is simply 2. 

For the TTF OIWFS, the average slope produced by the TT on
the quarter of the aperture $S_{\frac{1}{4}}=\frac{\pi}{4}$ (aperture units are normalised
by the aperture radius) is given by
\begin{align}\label{eq:pt}
  \gamma_\textsf{T}  & =
  \frac{1}{S_\frac{1}{4}}\int_0^1\int_0^{\sqrt{1-y^2}}
  \frac{\partial}{\partial x} Z_{2,3}
    (x,y) \partial x \partial y \\
    & = 2 \nonumber 
        \end{align}
The average slope produced by the focus mode is given by
\begin{align}\label{eq:pf}
  \gamma_\textsf{F}  & =
  \frac{1}{S_\frac{1}{4}}\int_0^1\int_0^{\sqrt{1-y^2}}
  \frac{\partial}{\partial x} Z_{4}
    (x,y) \partial x \partial y \\
    & = \frac{16\sqrt{3}}{3\pi} \nonumber 
        \end{align}
whereas for the astigmatisms 
\begin{align}\label{eq:pa}
  \gamma_\textsf{A}  & =
  \frac{1}{S_\frac{1}{4}}\int_0^1\int_0^{\sqrt{1-y^2}}
  \frac{\partial}{\partial x} Z_{5\cdots6}
    (x,y) \partial x \partial y \\
    & =  \frac{8\sqrt{6}}{3\pi} \nonumber 
  \end{align}
  The signal $\pm$ attached to
 $\gamma_\textsf{F}$ and $  \gamma_\textsf{A}$ in Eq. \eqref{eq:Gamma} is a function of
 the exact quadrant where each sub-aperture is located.

%------------section--------------------
\subsection{Noise model}

\subsubsection{Diffraction-limited case}
In the following, the noise model detailed in \cite{clare06} is
used. It is assumed that spots are diffraction limited. Therefore,
these equations apply for a Nyquist-sampled spot, \textit{i.e.} with
$2\times2$ pixels, by other words a quadrant detector.


The noise added to each sub-aperture measurement is given by (in angle
rms units)
\begin{equation}\label{eq:etab}
  \sigma_\eta = \frac{\theta_b}{\textsf{SNR}},   [rad]
\end{equation}
where $\theta_b$ is the effective spot size of the sub-aperture, and SNR is the signal-to-noise ratio of a
single sub-aperture. For a quadrant detector, the SNR
is given by
\begin{equation}\label{eq:SNR}
  \textsf{SNR}= \frac{N_p}{\sqrt{N_p+4N_b+4\sigma_e^2}},
\end{equation}
where $N_p$ is the number of photo-detection events per
sub-aperture, $N_b$ is the number of background photo-detection
events per sub-aperture, and $\sigma_e$ is the rms
detector read noise per pixel.

In the IR (H band), the NGS images are assumed to
contain a diffraction-limited core, for which case the effective
spot size is given by [Hardy Eq. 5.13]
\begin{equation}\label{eq:thetab}
  \theta_b = \frac{3\pi\lambda\sqrt{N_{sa}}}{16D_0},
\end{equation}
where $N_{sa}$ is the total number of sub-apertures for the
NGS WFS. The $2\times2$ NGS WFS is therefore noisier than any of the two
single sub-aperture NGS WFS. Note $\theta_b$ is twice that of the latter,
since $N_{sa}$ is 4 instead of 1 and that the number of
photo-detections per sub-aperture is also cut by a factor of 4 --
providing a $\sim$ 2 times smaller SNR. One thus ends up with a factor
$\sim$ 4
noisier measurement. The same reasoning applied to a general SH WFS
leads to standard estimations of the full-aperture gain. 

To convert to mas rms, multiply $\sigma_\eta [mas] = 
180/\pi \times 3600 \times 1000 \times \sigma_\eta [rad]$ 


\subsubsection{Seeing-limited case}
For the seeing-limited case, use instead for the photon noise

\begin{equation}
  \sigma^2_\eta = \frac{\pi^2}{2 ln(2)}\frac {1}{n_{ph}} \left(
    \frac{N_T}{N_D}\right)^2 [rad^2] ,
\end{equation}

where $N_T$ and $N_D$ are the FWHM of image and sub-aperture
respectively. With $N_T = \lambda/r_0$ and $N_D = \lambda/D$ one finds

\begin{equation}
  \sigma^2_\eta = \frac{\pi^2}{2 ln(2)}\frac {1}{n_{ph}} \left(
    \frac{D}{r_0}\right)^2 [rad^2] ,
\end{equation}
For the read-out noise we use
\begin{equation}
  \sigma^2_\eta = \frac{\pi^2}{3}\frac {\sigma_e^2}{n^2_{ph}} \left(
    \frac{N_S^2}{N_D}\right)^2 [rad^2] ,
\end{equation}
where $\sigma_e$ is the rms number of photoelectron event per pixel
and per frame, $N_S^2$ is the total number of pixels in the CoG computation. 


% If only photon-noise is accounted
% for, the SNR also decreases by a factor
% of 2 since $N_p$ is now a quarter of what it is for the
% single-aperture WFS, the same happening with the photo-detection
% events per pixel in each sub-aperture. For this case, the measurement
% noise on each of the  $2\times2$ NGS WFS sub-apertures is $4^2$ times the
% variance of the single-aperture case. Since there are 4 'x' and 4 'y'
% measurements, the total variance of the measurement error is hence
% $\sigma^2_{2\times2}=4 \sigma^2_{1\times1}$ for the TT-only...

% \begin{figure}[htpb]
%  	\begin{center}
% 	% use packages: array            
%          \begin{tabular}{cc}
%             \hspace{-20pt}\includegraphics[width=0.5\textwidth]{OIWFS_QuadCell_MeasError_nmrms_1x1_2x2subaps.eps}&  \hspace{-30pt}\includegraphics[width=0.5\textwidth]{OIWFS_QuadCell_RON.eps}
%        \end{tabular}
% 	\end{center}
% 	\caption[]
% 	{\label{fig:OIWFS_QuadCell_MeasError_nmrms_1x1_2x2subaps}
%           Left: Measurement noise error (photon-noise + read-out 
%           noise). Solid-lines: single sub-aperture OIWFS,
%           dashed-lines: 2x2 TTF OIWFS. The noise ratio in rms units between the $2\times2$ and the single
% sub-aperture OIWFS is $\sigma_{\eta, 2\times2}/\sigma_{\eta, 1\times1}
% \in \{4.04, \cdots, 7.96\}$ for $f_s \in \{20, \cdots, 800\}$Hz and
% $N_p \in\{5\times10^2, \cdots,5\times10^6\}$ph/m$^2$/s.
% \\ Right: The read-out noise is
%           $\sigma_e=\{2.99 \cdots 3.69\}$ for $f_s=\{20 \cdots 800\}$Hz.
% 	}
% \end{figure}


%%%%%%%%%%%%%%%%%%%%%%%%%%%%%%%%%%%%%%%% 
% ------------section--------------------
%%%%%%%%%%%%%%%%%%%%%%%%%%%%%%%%%%%%%%%% 
\section{Static reconstruction}
% ------------section--------------------

\subsection{Isoplanatic tilt correction}
A first approach consists in averaging the tilt measurements obtained
across the field 
\begin{align}
\Emv = \frac{1}{nGs}\sum_{i=1}^{nGs} \D^\dag{\svec}_{\alphavec,i} 
\end{align}
and/or eventually weigh the partial contributors by
$1/\sigma^2_{\eta,i}$, i.e. the inverse of the noise variance level on
each direction.

\subsection{Tilt tomography using virtual DMs}
The classical approach to derive tilt anisoplanatism is to consider
that field-dependent tilt is a linear combination of aperture-plane
tilt with pupil and high-altitude quadratic modes of
opposite signs whose resultant through ray-racing produces pure tilt
(the quadratic terms vanish) \cite{flicker02, ellerbroek01}. A similar approach can still be used in
laser tomography provided the quadratic terms are scaled by the cone
shrinking factor -- this is the general case shown next. At least 3
independent measurements of tilt are required in the field (i.e. 6
measurements) from which 2 tip/tilt and 3 quadratic modes
(differential focus and astigmatisms) are estimated. In real systems,
since the LGS cannot measure focus neither due to Na-range
fluctuations, one of the NGS WFSensors is actually a 2x2 WFS
providing focus and astigmatism measurements. We thus end up with a
total of 12 measurements to estimate 6 modes -- which is somehow
sub-optimal since we're not making use of the full information
contained in the measurements -- as was done more recently in
\cite{gilles11a}.

In the following we assume a split-tomography framework
\cite{gilles08a} and follow on the footsteps and notation of \cite{correia13}. 

For natural
tomography AO (with tilt-removed high-order WFS measurements) is can
be computed from what follows with ease. 


Following the above considerations the resulting
aperture-plane wave-front (WF) is conveniently expanded onto a
truncated  orthonormal Zernike polynomial's basis 
defined over 2
layers. 
For the
NGS modes model, only modes $Z_{2\cdots 6}$ are used that
correspond to the TT and the quadratic modes of
Eq. \eqref{eq:phi_aggregate_PS_modes}.%  where pure focus is already included, \textit{i.e.} $ {\varphi}^\tur _{k} $ is a
% 10-coefficient column vector.

The total aperture-plane wave-front error induced by the TT/TA modes at time instant
$t$, in direction $\boldsymbol\theta$ is given by
\begin{equation}\label{eq:phi_aggregate_PS_modes}
W(\boldsymbol{\rho}, \boldsymbol{\theta}, t) = \sum_{i=2}^{6}
a_i(t)Z_i\left(\frac{\boldsymbol\rho}{R_0}\right)
 + \sum_{i=2}^{6}
b_i(t)Z_i\left(\frac{\boldsymbol\rho + \boldsymbol{\theta} h_\text{\tiny{DM}}}{R_h}\right)
\end{equation}
where $\boldsymbol\rho $ is a two-dimensional space coordinate vector and
$\mathbf{a}$ and $\mathbf{b}$ are vectors containing the Zernike coefficients (following
the ordering of \cite{noll76}) defined over the lower and upper DM-conjugate
planes of radius $R_0=D_0/2$ and $R_h = D_h/2$ % -- see
% Fig. \ref{fig:ps_scheme}.
Note $i \in \{2,\cdots,6\}$ which
effectively represents 5 modes; $i=1$ refers to piston and is
disregarded as this mode has no impact on AO performance. 
Global TT have the corresponding Zernike
modes applied to the ground DM only, \textit{i.e.} $b_2 = b_3=0$. 

The coefficients $b_i$ can be worked out such that 
$W(\boldsymbol{\rho}, \boldsymbol{\theta},t)$ only produces TT in the LGS WFSs (but
not on the NGS WFSs). 
This constraint provides  $b_i = -
r_l^{-2} a_i$, with $r_l$ given by the ratio of the
cone-intersected pupil and meta-pupil in the DM conjugate range is 
\begin{equation}\label{eq:r_l}
  r_l %\egrefs{eq:r_p}{eq:r_n} 
= r_n \underbrace{\left(1-\frac{h_\text{\tiny{DM}}}{h_{\text{\tiny{Na}}}}\right)}_{r_c},
\end{equation}
where $  r_c \triangleq1-h_\text{\tiny{DM}}/h_{\text{\tiny{Na}}}$
% \begin{equation}\label{eq:r_p}
%   r_c \triangleq1-h_\text{\tiny{DM}}/h_{\text{\tiny{Na}}}
% \end{equation}
is the shrinking factor of the cone-intersected
meta-pupil diameter with respect to the cylinder-intersected
meta-pupil diameter at
$h_\text{\tiny{DM}}$\,km, translating the cone effect for a DM conjugated to
range $h_\text{\tiny{DM}}$ and an LGS at range $h_{\text{\tiny{Na}}}$ and  
\begin{equation}\label{eq:r_n}
          r_n \triangleq \frac{D_0}{D_0+FoV\times h_\text{\tiny{DM}}\times 1000},
        \end{equation}
normalizes 
the upper modal coefficient to the particular choice of underlying meta-pupils over which the modes are defined, with
$FoV$ the field-of-view in radians, $h_\text{\tiny{DM}}$ the conjugate
altitude of the upper DM in km. 




Setting $b_i = -
r_l^{-2}\alpha_i$ in Eq. \eqref{eq:phi_aggregate_PS_modes} and
solving for the aperture-plane field-dependent TT % -- $\TT(\boldsymbol{\rho}, \boldsymbol{\theta},t)$ -- is found from \cite{flicker03}
\begin{equation}\label{eq:field-dependent-TT}
\TT(\boldsymbol{\rho}, \boldsymbol{\theta},t)= \sum_{i=2}^{3}
\xi_i(\mathbf{\theta}, t)Z_i\left(\frac{\boldsymbol{\rho}}{R_0}\right)
\end{equation}
with the field-dependent TT coefficients
$\boldsymbol\xi(\boldsymbol\theta,t)$ \cite{flicker03} 
\begin{align}\label{eq:Z4h_tip_tilt}
            \boldsymbol\xi(\boldsymbol\theta,t)
& =  %  \left( 
%             \begin{array}{ccccc}
%               r_n&0&r_n^2\frac{2\sqrt{3}h\theta_x}{R} &
%               r_n^2\frac{\sqrt{6}h\theta_y}{R} & r_n^2\frac{\sqrt{6}h\theta_x}{ R}\\ 
% 0&r_n&r_n^2\frac{2\sqrt{3}h\theta_y}{R} &
%               r_n^2\frac{\sqrt{6}h\theta_x}{R} & - r_n^2\frac{\sqrt{6}h\theta_y}{ R}
%               \end{array}\right) 
%         \ProjTheta(\theta)    \Proj_{\QPS} \ps(t)\\ 
% &
% = 
\left( 
            \begin{array}{cccccc}
              1&0&-r_c^{-2}\frac{2\sqrt{3}h\theta_x}{R_0} &
              -r_c^{-2}\frac{\sqrt{6}h\theta_y}{R_0} &
              -r_c^{-2}\frac{\sqrt{6}h\theta_x}{ R_0} &0 \\ 
0&1&-r_c^{-2}\frac{2\sqrt{3}h\theta_y}{R_0} &
              -r_c^{-2}\frac{\sqrt{6}h\theta_x}{R_0} &
              r_c^{-2}\frac{\sqrt{6}h\theta_y}{ R_0} &0
              \end{array}\right) \boldsymbol\ps(t)
\end{align}
In Eq. \eqref{eq:Z4h_tip_tilt}
$\boldsymbol\ps(t)\in \Re^{6\times1}$
is a vector with the coefficient of the NGS modes (numerically equal
to $\mathbf{a}$ in Eq. \eqref{eq:phi_aggregate_PS_modes} but not to be
confounded with it) plus a sixth coefficient to model pure focus
induced by  $N_a$-layer that does not produce TT.

Expressing the Zernike polynomials using cartesian coordinates one gets
\begin{subequations}\label{eq:TT_PS_definition}
\begin{align}\label{eq:plate_scale_def}
  \xvec_{k}^{\textsf{tip}} & = 2(x_0, 0);\\
   \xvec_{k}^{\textsf{tilt}}  & = 2(y_0, 0);\\
   \xvec_{k}^{\Delta F}  & = \sqrt{3}([x_0^2 + y_0^2]; -[x_c^2 + y_c^2]/r_l^2);\\
   \xvec_{k}^{\Delta A_0} & = \sqrt{6} ([x_0^2 - y_0^2]; -[x_c^2 - y_c^2]/r_l^2);\\
   \xvec_{k}^{\Delta A_{45}}  & = \sqrt{6} ([x_0 y_0];-[x_c y_c]/r_l^2);
\end{align}
\end{subequations}
where $x_0$, $y_0$ and $x_c$, $y_c$ are the actuator coordinates
on the ground and upper DMs, normalised by the meta-pupil radius at 0
and $h_\text{DM}$.


The coefficients $\boldsymbol\ps(t)$ can be found from least-squares
fitting the tilt and quadratic modes to the atmosphere (in passing, this is by no means different
from the DM fitting step to an estimated tomographic phase -- in the
latter the DM influence functions are used instead of the quadratic
modes defined previously).
\begin{equation}
  \boldsymbol\ps = \argmin_{\boldsymbol\ps} \average{\vert
    \Proj_\theta \varphivec - \Proj_\theta^\ps \Proj_{\QPS}\boldsymbol\ps  \vert^2}_\text{FoV}
\end{equation}
where
\begin{align}
  \phivec & = \Proj_\theta \varphivec
\end{align}
is the pupil-plane integrated wave-front profile in direction
$\thetavec$ and 
\begin{align}
 \boldsymbol\xi(\boldsymbol\theta,t) & = \Proj_\theta^\ps
                                       \Proj_{\QPS}\boldsymbol\ps (t)
\end{align}
is the tip/tilt produced by the TT/TTA modes on those directions.

In addition, 
\begin{equation}\label{eq:Pq2ps}
          \Proj_{\QPS}\triangleq \left( 
            \begin{array}{ccccc:c}
              1 & \0 & \0 & \0 & \0 &\0\\ 
             0 & 1 & \0 & \0 & \0 &\0\\ 
             0 & \0 & 1 & \0 & \0 &1\\ 
          0 & \0 & \0 & 1 & \0 &\0\\  
          0 & \0 & \0 & 0 & 1 &\0\\  \hdashline
          0 & \0 & \0 & 0 & \0 &\0\\  
          0 & \0 & \0 & 0 & \0 &\0\\  
          0 & \0 & -1/r_l^2 & 0 & 0 &\0\\  
          0 & \0 & \0 & -1/r_l^2 & \0 &\0\\  
          0 & \0 & \0 & 0 & -1/r_l^2 &\0\\  
           \end{array}%
          \right),
        \end{equation}
providing the 2-layer Zernike modal decomposition of the WF from the
NGS modal coefficients.

By setting $\boldsymbol{M}\triangleq \Proj_\alpha^\ps \Proj_{\QPS}$ one
then gets 
\begin{equation}
  \boldsymbol\ps(t) = \average{\boldsymbol{M}^\T
    \boldsymbol{M}}_\text{FoV}^{-1}
  \average{\boldsymbol{M}^\T\Proj_\alphavec}_\text{FoV} \varphivec(t)
\end{equation}
In practice this is done over a few directions in the FoV with weights
applied on them (e.g. Simpson weights).
% In the current NFIRAOS design, the NGS control loop uses a noise-weighted, least square reconstructor and can
% operate at a different frame rate from the LGS
% loop depending upon the brightness of the NGS \cite{wang09}.

Using the measurement equation
\begin{equation}\label{eq:fwd_MOAO_meas_model}
\svec_\alphavec(t) = \D \Proj_\alphavec\varphivec (t) + \etavec(t)
\end{equation}
in which we project directly from the two virtual layers, the tilt can
be estimated from the noise-weighted estimator
\begin{equation}
\Emv =   \Proj_\betavec\left(\Proj_\alphavec^\T\D^\T\CovMat_\eta^{-1} \D \Proj_\alphavec\right)^{-1}\Proj_\alphavec^\T\D^\T\CovMat_\eta^{-1}
\end{equation}

%%%%%%%%%%%%%%%%%%%%%%%%%%%%%%%%%%%%%%%% 
% ------------section--------------------
%%%%%%%%%%%%%%%%%%%%%%%%%%%%%%%%%%%%%%%% 
% \subsection{ Focus error due to $N_a$-layer range variations}\label{sec:n_a-layer-altitude-variations}

Although the NGS modes
produce only field-dependent TT not seen by the LGS WFS, for a NGS
looking upwards through a cylinder and not a cone, focus and
astigmatisms are indeed probed.


\subsection {Tilt tomography with spatio-angular reconstruction in NTAO and LTAO}

In the NTAO\footnote{Assume tilt-removed measurements.}/LTAO case, the tilt compensation is greatly simplified
since an explicit tomographic estimate of tilt is not mandatory. We
thus fall in the spatio-angular case whereby the computation of
field-dependent covariances is done off-line and applied directly
on-line. 


We now develop the  \textit{minimum mean square error} (MMSE) solution with a simplified measurement model involving the pupil-plane turbulence only
\begin{equation}\label{eq:fwd_MOAO_meas_model}
\svec_\alphavec(t) = \D \phivec_\alphavec (t) + \etavec(t)
\end{equation}
Assuming $\svec$ and $\phivec$ are zero-mean and jointly Gaussian, direct application of the MMSE solution to estimate the aperture-plane
phase in the $N_\beta$-science directions of interest yields \cite{andersonmoore_optimalfiltering05} 
\begin{align}\label{eq:MMSE_theor_MOAO}
\mathcal{E}\{\phivec_\betavec|\svec_\alphavec\}  \triangleq 
\CovMat_{(\phivec_\betavec,\svec_\alphavec)}\CovMat^{-1}_{\svec_\alphavec} {\svec}_\alphavec = \widehat{\phivec}_\betavec %= \Emv {\svec}_\alphavec
\end{align}
where $\mathcal{E}\{X|Y\} $ stands for mathematical expectation of $X$
conditioned to $Y$.  Since in general $\betavec \neq \alphavec$,
Eq. (\ref{eq:MMSE_theor_MOAO}) follows from
$\mathcal{E}\{\phivec_\betavec|\svec_\alphavec\} =
\mathcal{E}\{\phivec_\betavec|
\mathcal{E}\{\phivec_\alphavec|\svec_\alphavec\} \}$.  Matrices
$\CovMat_{(\phivec_\betavec,\svec_\alphavec)}$ and
$\CovMat}_{\svec_\alphavec}$ are
spatio-angular covariance matrices that relate the tilt on direction
$\theta$ to that on direction $\beta$. These matrices are computed
using formulae in \cite {whiteley98a} from the simulator OOMAO \cite {conan14}.

Under the spatio-angular approach, it is also possible to compute
$\phivec(\rhovec, \betavec, t+\Delta)$, i.e., temporally predict the TT
ahead in time by adjusting the angles over which the correlations are
computed \cite {correia15}.


Given that the conditioning relates only to the very last available measurement (as opposed to present and previous measurements), these reconstructors are labelled as \textit{static}.
Developing terms in Eq. \eqref{eq:MMSE_theor_MOAO} using Eq. (\ref{eq:fwd_MOAO_meas_model})
the reconstructor becomes 
\begin{equation}\label{eq:hat_phi_Collapsed}
\widehat{\phivec}_\betavec \triangleq\average{ \phivec_\betavec \phivec_\alphavec^\T }
\D^\T \left(\D \average{ \phivec_\alphavec \phivec_\alphavec^\T } \D^\T +
  \average{ \etavec \etavec^\T }\right)^{-1} {\svec}_\alphavec
\end{equation}

This
MMSE reconstructor is dubbed \textit{spatio-angular} (SA) reconstructor on
account of the nature of the covariance matrices involved in its
definition. It can be seen as a generalisation of the work of Whitely
\textit{et al} \cite{whiteley98a} in seeking the optimal anisoplanatic
reconstuctor in classical AO to the tomographic, multiple sensor case.
It has several convenient features: is much faster to compute off-line
than the \textit{explicit tomography} reconstructor for large modal sets and circumvents the truncated expansion on a modal basis. 

When phase is expanded onto the Zernike polynomials the spatio-angular cross-covariance functions can be analytically computed
 for the infinite outer-scale case of turbulence \cite{valley79, chassat92}
\begin{equation}\label{eq:CalphaZernike}
\average{\phi_i(0) \phi_j(\xi)} =
  3.895\left(\frac{D}{r_0}\right)^{\frac{5}{3}}
  \frac{\int_0^{h_\text{max}} C_n^2(h)I_{ij}\left(\frac{\xi
          h}{R}\right) \dint h}{\int_0^\infty C_n^2(h) \dint h}
\end{equation}
with $D=2R$ the telescope diameter, $r_0$ the Fried parameter, $h$ the
altitude $\xi$ the angle between the pupils over which
the Zernike polynomials are defined, $C_n^2$ the atmospheric vertical profile and $I_{ij}(x)$ a term involving 1-D numerical integration. 
% \begin{align}\label{eq:Iij}I_{ij}(x) = & (-1)^{\frac{n_1+n_2-m_1-m_2}{2}} \sqrt{(n_1+2)(n_2+1)}
% \times \nonumber\\
% & \left[ K_{i,j}^+ \int_0^{\infty} \kappa^{-\frac{14}{3}}
%   J_{n_i+1}(2\pi\kappa) J_{n_{j}+1}(2\pi\kappa) J_{m_i+m_j}(2\pi\kappa
%   x) \dint \kappa \right. \nonumber\\
% & \left. + K_{i,j}^- \int_0^\infty \kappa^{-\frac{14}{3}}
%   J_{n_i+1}(2\pi\kappa) J_{n_j+1}(2\pi\kappa) J_{|m_i-m_j|}(2\pi\kappa
%   x) \dint \kappa\right]
% \end{align}
% with $K_{i,j}^+$ and $K_{i,j}^-$ coefficients that depend on $m_i$ and
% $n_i$.
% For the general case, take $\xi$ to be the angular difference of
% $\alpha_i$ and $\alpha_j$.
Equation (\ref{eq:CalphaZernike}) has been extended for the finite
outer-scale case in \cite{takato95} and later extensively used and
generalised in \cite{whiteley98, whiteley98a}. The layered spatial
covariance matrix $\average{ \varphivec \varphivec^\T}$ is block-diagonal (layers are independent) and can be found in \cite{noll76} for 
the infinite outer scale case and in \cite{winker91} for the finite case.




%------------section--------------------
\subsection{Review of reconstructors }
- MMSE

\begin{align}
\Emv = 
\CovMat_{(\phivec_\betavec,\svec_\alphavec)}\CovMat^{-1}_{\svec_\alphavec} {\svec}_\alphavec 
\end{align}
- GLAO-like
\begin{align}
\Emv = \frac{1}{nGs}\sum_{i=1}^{nGs} \D^\dag{\svec}_{\alphavec,i} 
\end{align}
- virtual DM on two layers

\begin{align}
\Emv = \Proj_\betavec\left(\Proj_\alphavec^\T\D^\T\CovMat_\eta^{-1} \D \Proj_\alphavec\right)^{-1}\Proj_\alphavec^\T\D^\T\CovMat_\eta^{-1}
\end{align}
This reconstructor is a noise-weighted reconstructor. In case of
equally noisy measurements it boils down to the simple averaging
(GLAO-like) case. 



\subsection{Performance assessment}

For the MMSE case, the standard performance assessment formulae
apply. 

Starting from
\begin{equation}\label{eq:res_var}
\sigma^2(\betavec) = \average{\left\Vert \phivec(\betavec) - \widehat{\phivec}(\betavec)\right\Vert^2}
\end{equation}
with $\widehat{\phivec}(\betavec) = \Rec  {\svec}(\alphavec)$ we get

\begin{align}\label{eq:res_var}
\sigma^2(\betavec) & = \tr\{\CovMat_\betavec\} +
\tr\{\Rec\D\CovMat_\alphavec\D^\T\Rec^\T\}  \\ & -
\tr\{\Rec\D\CovMat_{\betavec,\alphavec}\}  -
\tr\{\CovMat_{\alphavec, \betavec}\D^\T\Rec^\T\}  \\ & + \tr\{\Rec\CovMat_\etavec\Rec^\T\} 
\end{align}

We can as well account for temporal errors by estimating instead the
following 
\begin{equation}\label{eq:res_var_time}
\sigma^2(\betavec, T_s) = \average{\Vert \phivec(\betavec,t+T_s) - \phivec(\betavec,t)\Vert^2}
\end{equation}
leading to similar expressions as before but where a lag equal to
$T_s$ is considered when computing the integrals with a displacement
between the meta-pupils that increases by ${v}T_s$ with
$\mathbf{v}$ the wind velocity vector ($v = |\mathbf{v}|$).

A more sound way to compute the residuals is by considering still
Eq. \eqref{eq:res_var} when the reconstructor has built-in prediction 
\begin{equation}\label{eq:res_var_time}
\Rec =
\CovMat_{(\phivec_{\betavec,k+1},\svec_{\alphavec,k})}\CovMat^{-1}_{\svec_{\alphavec,
k} }
\end{equation}
and computing 
\begin{align}\label{eq:res_var}
\sigma^2(\betavec) & = \tr\{\CovMat_\betavec\} +
\tr\{\Rec\D\CovMat_\alphavec\D^\T\Rec^\T\} \\ & -
\tr\{\Rec\D\CovMat_{(\betavec,k+1),(\alphavec,k)}\}  -
\tr\{\CovMat_{(\alphavec,k), (\betavec,k+1)}\D^\T\Rec^\T\}  \\ & + \tr\{\Rec\CovMat_\etavec\Rec^\T\} 
\end{align}
where I omit the temporal dependence whenever possible due to stationarity.

\section{ Dynamic controller of NGS modes}
We'll use the approach laid out in \cite{correia15} in which the
states are directly the pupil-integrated TT modes in the GS directions
coupled to a state-transition model using a near-Markovian
time-progression model. We can thus include information from a large
number of layers but never explicitly estimate those layered modal
contributions to the final NGS modes in the pupil in the GS
directions. 

The angular cross-covariance matrices give rise to order-1 Markovian
models with non-null temporal cross-spectra (iff more that TT are considered). A comparison to decoupled
high-order models (AR for instance) for the NGS-only modes must be
conducted. A temporal TF analysis is most welcome -- if this problem
is tractable. 

\subsection{near-Markovian time-progression model of the 1st-order }\label{sec:nearMarkovModel}
The use of auto-regressive models has been given wide attention using
the Zernike polynomials' expansion basis set \cite{petit09, sivo14} as
well as with Fourier modes for which a complex model encodes perfectly
spatial shifts and thus frozen-flow~\cite{poyneer08}. %Several time-progression models have been analysed over the past few years. % Predictive schemes exploring the frozen-flow are most common either embedding layer disentanglement \cite{poyneer08} or not \cite{lukin10}.

We restrict our attention to the near-Markovian first-order
time-evolution model \cite{gavel02a} which has delivered best overall
performance albeit with increased computational
complexity~\cite{piatrou07, correia14, jackson15} (spatial dependence omitted)
\begin{equation}\label{eq:AR1model}
  \phivec_{k+\Delta}(\alphavec) = \Asa \phivec_{k} (\alphavec) + \boldsymbol{\nu}_k (\alphavec)
\end{equation}  
with  $\Delta = \tau/T_s$ is a delay in units of sample step $T_s$, where the transition matrix minimizes the quadratic criterion \cite{correia14}
\begin{equation}\label{eq:Asa}
 \Asa = \arg\min_{\Asa'}\average{\left\Vert\phivec_{k+\Delta} (\alphavec)- \Asa'  \phivec_k(\alphavec) \right \Vert^2_{L_2(\Omega)}},
\end{equation}

The solution to Eq. (\ref{eq:Asa}) is 
\begin{equation}\label{Asa_Def}
\Asa = \average{\phivec_{k+\Delta}\phivec_k^\T } \average{ \phivec_k \phivec_k^\T }^{-1} (\alphavec)
\end{equation}
Note that we do not remove piston in this equation although we could following formulae in \cite{wallner83}.

Assuming stationarity, the state excitation noise covariance matrix is found for a first order time-evolution model
from the covariance equality (implicit indices dropped out)
 \begin{equation}\label{eq:state_noise_matrix}
   \average{\phivec \phivec^\T } = \Asa
   \average{ \phivec \phivec^\T }
   \Asa^\T + \average{\boldsymbol{\nu} \boldsymbol{\nu}^\T},
 \end{equation}
since $\average{\phivec_{k+1}\phivec_{k+1}^\T } = \average{ \phivec_{k} \phivec_{k}^\T } = \average{\phivec_{}\phivec_{}^\T }$. The excitation noise covariance matrix is therefore
\begin{equation}
\average{\boldsymbol{\nu} \boldsymbol{\nu}^\T} = \average{ \phivec \phivec^\T } - \Asa \average{ \phivec \phivec^\T } \Asa^\T
 \end{equation}  
The model driving noise covariance matrix $\average{\boldsymbol{\nu} \boldsymbol{\nu}^\T}$ is a key element of the KF design.

\subsection{State-space representation}

In the spatio-angular framework, we will make use of a state vector with
the pupil-plane phase integrated over directions $\alphavec \in[1, \cdots, N_\alpha]$ at instant $k$ to provide the \textit{Spatio-Angular} LQG formulation. 

Selecting thus $\xvec_{k}\triangleq \phivec_{k}(\alphavec)$ 
\begin{equation}
\phivec_{k}(\alphavec) =
\left(\begin{array}{c}
 \phivec({\alphavec_1})\\ \cdots\\ \phivec({\alphavec_{N_\alpha}})
\end{array}\right)_k
\end{equation}
where the state is a concatenation of phase vector in the $N_\alpha$ guide-star directions, 
one defines the state space terms for Eq. (\ref{eq:AR1model})
\begin{equation}\label{eq:stack_ss_c_tur_compact}
  \left[\begin{array}{c|c|c}
     \Ad & \Bd &  \Vd\\\hline
      \Cd & \Dd  & \0
    \end{array}\right]
=
\left[\begin{array}{c|c|c}
      \Asa & \0 & \I \\\hline
      \D & -\D\z^{-d} & \0
    \end{array}\right]
\end{equation}

The implementation of the KF involves a real-time state update and prediction equations which in the SA case are
\begin{subequations}\label{eq:single_rate_StatSA_RToperations}
\begin{align}
  \widehat{\phivec}_{k|k}(\alphavec) & = \widehat{\phivec}_{k|k-1}(\alphavec) + \mathcal{H}_\infty\left(\svec_k(\alphavec) -
     \D \widehat{\phivec}_{k|k-1}(\alphavec) +  \D \uvec_{k-d}\right) 
\\
  \widehat{\phivec}_{k+1|k}(\alphavec) & = \Asa\widehat{\phivec}_{k|k} (\alphavec)
\end{align}
\end{subequations}
where $\mathcal{H}_\infty$ is the asymptotic Kalman gain computed from the solution of an estimation Riccati equation \cite{correia10a}. The use of the asymptotic value is justified, like elsewhere \cite{petit08} for one seeks long-exposure performance therefore employing the steady-state gain with no loss of performance. Whenever implicit we drop the notation $(\alphavec)$.


% ------------section--------------------
\subsection{LQG + time-smoothing operation}


For low frame-rates when guiding on dim stars replacing the central point estimate by the average over tha sampling interval may lead to further performance gains. There should be no added real-time complexity added, only off-line extra computation. 

This up-sampling model may be required anyway one the (way less computationally intensive) TT vibration modes, following \cite{correia12}.


The command up-sampling could be easily evaluated with the SA algorithm, since one could estimate 
\begin{align}
  \uvec_{k+\Delta} & = \mathbf{F} \CovMat_{\phivec_\betavec,\phivec_\alphavec}\CovMat_{\phivec_\alphavec,\phivec_\alphavec}^{-1} \mathcal{A}_\Delta\widehat{\phivec}(\alphavec)_{ k+1|k}\\
\end{align}
could be replaced by 
\begin{align}
  \uvec_{k+\Delta} & = \mathbf{F} \mathcal{A}'_\Delta\CovMat_{\phivec_\betavec,\phivec_\alphavec}\CovMat_{\phivec_\alphavec,\phivec_\alphavec}^{-1} \widehat{\phivec}(\alphavec)_{ k+1|k}\\
& = \mathbf{T}'_\Delta\widehat{\phivec}(\alphavec)_{ k+1|k}
\end{align}
i.e. first estimate the phase in the $\betavec$ direction and then make the temporal prediction. 

However, this raises a question that has not yet been addressed: the phase estimate is an approximation of the average at the middle of the integration interval. The actual average phase could be computed (at least numerically) by doing
\begin{align}\label{eq:u_avr}
  \uvec_{k+\Delta} & = \mathbf{F} \overline{\mathcal{A}}'_\Delta\CovMat_{\phivec_\betavec,\phivec_\alphavec}\CovMat_{\phivec_\alphavec,\phivec_\alphavec}^{-1} \widehat{\phivec}(\alphavec)_{ k+1|k}
\end{align}
with 
\begin{align}\label{eq:A_avr}
  \overline{\mathcal{A}}'_\Delta & \triangleq \frac{1}{M}\sum_{i=1}^{M-1} \mathcal{A}_{\Delta + i/M} \\
& = \frac{1}{M}\sum_{i=0}^{M-1} \left\{\average{\phi_{k+\Delta }\left[\betavec - \mathbf{v} (-T_s/2 + i/M T_s) \right]\phi_k^\T } \right\} \average{\phi_{k}\phi_k^\T }^{-1} \\
& = \frac{1}{M}\sum_{i=0}^{M-1} \left\{\average{\phi_{k}\left[\betavec - \mathbf{v} (\Delta/T_s -T_s/2 + i/M T_s) \right]\phi_k^\T } \right\} \average{\phi_{k}\phi_k^\T }^{-1}
\end{align}
which can probably be further simplified. % $M = T_s/2ms$ is the integer ratio of the integration time over the 2ms Raven fastest DM up-date frame-rate. 

The time-averaged phase is
\begin{align}
  \overline{\phivec}_k = \frac{1}{T_s}\int_{-T_s/2}^{T_s/2}{\phivec}(kT_s+t) \dint t = 
\overline{\mathcal{A}}'{\phivec}_k 
\end{align}
where 
\begin{align}
  \overline{\mathcal{A}}' & =  \overline{\average{\phi_{k+\Delta}\phi_k^\T } } \average{\phi_{k}\phi_k^\T }^{-1}
\end{align}
where $\overline{\average{\phi_{k+\Delta}\phi_k^\T } }$ is computed in a single-step from $\overline{C}(\boldsymbol{\rho})$ using the averaged covariance function
% \begin{equation}
%   \overline{C}(\boldsymbol{\rho},T_s) = \frac{1}{T_s}\int_{-T_s/2}^{T_s/2}C(\boldsymbol{\rho} - \mathbf{v} t) \dint t
% \end{equation}
\begin{equation}
\overline{C}(\rhovec, T_s) = \sum_{l=1}^L \omega_l \frac{1}{T_s}\int_{-T_s/2}^{T_s/2} C(\boldsymbol{\rho} - \mathbf{v} t) \dint t
\end{equation}
or by averaging multiple instances of the covariance matrices as done in Eq. (\ref{eq:A_avr}).

With this smoothed model one can instead have, still with  $ \xvec_{k}\triangleq \phivec_{\alphavec,k}$ 
\begin{equation}\label{eq:stack_ss_c_tur_compact}
  \left[\begin{array}{c|c|c}
      \Ad & \Bd &  \Vd\\\hline
      \Cd & \Dd  & \0
    \end{array}\right]
=
\left[\begin{array}{c|c|c}
      \mathcal{A}_\alphavec & \0 & \I \\\hline
      \D\overline{\mathcal{A}}' & -\D\z^{-d} & \0
    \end{array}\right]
\end{equation}
with the smoother commands given in Eq. (\ref{eq:u_avr}) and $\D\overline{\mathcal{A}}'$ a matrix that produces the WFS gradients from the averaged wave-front over the integration period $T_s$. Matrix $\overline{\mathcal{A}}'$ approaches identity as the integration time decreases.

\subsection{LQG + multi-rate}

It has been show (unpublished though) that the multi-rate case cannot
be exploited in the TT measurement out of 2 TT and 1 TTFA SH-WFS for
there are only 6 independent vectors to estimate 5 unknowns (2 TT and
3 plate-scale modes). 

Multi-rate could be potentially applied if one of the measurements
would provide more modes than just linear and quadratic. This could
eventually be the case for TT measured close to on-axis in NIR with
partial correction by the LGS loop. In that case LIFT would operate
and the multi-rate controller can potentially be further explored.


\section{Split tomography v. integrated tomography}
To be written later.

{\color{red} 
We need here the block diagrams for the spit and integrated tomography}



\section{Design of HARMONI (E-IFU) tilt anisoplanatism controller}
To be written later.

\subsection{Isoplanatic tilt}
Wind buffeting on the telescope

\subsection{Anisoplanatic tilt}
Atmospheric tilt

\subsection{Non-stationary tilt from M4 off-loading}
The famous "schlonk" every 5 minutes or so.
\appendix

% %------------section--------------------
\section{Average-slope removal from measurements}
For E2E simulations , the average slope removal is done in gradient space by projecting off
the average x and y slopes. 
\begin{align}
{s}_{ol,TTR} & = s_{ol} - MM_\eta^\dag  s_{ol}\\
             & = (\I - MM_\eta^\dag)  s_{ol}
\end{align}
where 
\begin{align}
M & = \left[\begin{array}{ll}
\mathbf 1 & \mathbf 0\\
\mathbf 0  & \mathbf 1
\end{array}\right]
\end{align}
a matrix with size of slope measurements by 2 (modes) and
\begin{align}
M_\eta^\dag & = (M^\T \CovMat_\eta^{-1} M)^{-1}M^\T \CovMat_\eta^{-1}
\end{align}
the noise-weighted pseudo-inverse of $M$ with $ \CovMat_\eta^{-1}$ the
inverse measurement noise covariance matrix.


      % ------------section--------------------
        \section{LQG controller transfer functions}\label{sec:fonct-de-transf-LQG}

       \begin{property} \textbf{LQG transfer functions.}
   
     Let the general transfer
          \begin{equation}
            \mathbf{y}(\z) = \mathbf{C}(\z) \mathbf{u}(\z).
          \end{equation}
          For a LQG using the predictor version $\uvec_k =
          -\K \widehat{\xvec}_{k+1|k}$
          the transfer function $\mathbf{C}_p(\z)$
          is
          \begin{equation}
            \label{eq:Kalman_predictor_TF}
            \mathbf{C}_p(\z)  =
            - \left( \I + \K \Lambda_{p} \z^{-d} \Lkal \mathcal{D} \right)^{-1} \K 
            \Lambda_{p} \Ad \Hkal,
          \end{equation}
          with 
           \begin{equation}
            \Lambda_{p} = \left(\I - \z^{-1} \left( \Ad -  \Lkal \Cd
              \right)\right)^{-1}.
          \end{equation}
         
If instead we were to compute   $\uvec_k = -\K \widehat{\xvec}_{k|k-1}$ then
one would get
  \begin{equation}
    \mathbf{C}_p(\z) =  - \left[ \I + z^{-1}\K \Lambda_{p} \left(\z^{-d} \Lkal
      \mathcal{D} + \Bd \right)\right]^{-1} z^{-1}\K
            \Lambda_{p} \Ad \Hkal  .
  \end{equation}

  For a command $\uvec_k = -\K
  \widehat{\xvec}_{k|k}$,

 \begin{equation}
    \mathbf{C}_e(\z)  = - \left\{\I + \K \Lambda_{e} \left[z^{-1}
        (\I+\Hkal\Cd) \Bd + \z^{-d}\Hkal\mathcal{D}\right] \right\}^{-1} \K \Lambda_{e} \Hkal,
  \end{equation}

where 
  \begin{equation}
    \Lambda_{e} = \left(\I - \z^{-1}\Ad + \z^{-1} \Hkal \Cd \Ad \right)^{-1}.
  \end{equation}

\propend
        \end{property}

        \begin{demo} \textbf{Demonstration.}

          The state-update and state-estimate equations of the LQG are
          \begin{align}\label{eq:Kalman_estimation_update}
            \left\{\begin{array}{cl}
                \widehat{\xvec}_{k|k} & = \widehat{\xvec}_{k|k-1} + \Hkal (\svec_k - \Cd
                \widehat{\xvec}_{k|k-1} + \mathcal{D} \uvec_{k-2} ) \\
                \widehat{\xvec}_{k+1|k} & = \Ad \widehat{\xvec}_{k|k} + \Bd \uz_k
              \end{array}\right. ,
          \end{align}
          The second line of Eq.~(\ref{eq:Kalman_estimation_update})
          can be likewise written, using $\widehat{\xz}_p$ and
          $\widehat{\xz}_e$ the $\mathcal{Z}$-transfers of $\widehat{\xvec}_{k|k}$ and $\widehat{\xvec}_{k|k-1}$,
          \begin{equation}
            \z \widehat\xz_{p}  = \Ad \widehat{\xz}_{e} + \Bd \uz.
          \end{equation}
          I assume next that the state cannot be attained by the
          commands, i.e. $\Bd = \0$.

          Multipying out by $z^{-1}$ on both sides, one gets 
          \begin{equation}
            \widehat{\xz}_{p} = \z^{-1} \Ad \widehat{\xz}_{e}.
          \end{equation}
          With this result and the second line in Eq.~(\ref{eq:Kalman_estimation_update})
          \begin{align}
            \widehat{\xz}_{p} & = \z^{-1} \Ad \left(\widehat{\xz}_{p} +
              \Hkal \left( \yz - \Cd \widehat{\xz}_{p}  + \z^{-2}
                \mathcal{D} \uz\right)\right) \nonumber \\
            %& = \z^{-1} \Ad \left( \I - \Hkal \Cd \right)
            %\widehat{\xz}_{k|k-1} + \z^{-1} \Ad \Hkal 
            %\yz_k + \z^{-1} \Bd
            %\uz_k\\
            & = \left(\I -  \z^{-1}\Ad \left( \I - \Hkal \Cd
              \right)\right)^{-1} \z^{-1} \Ad \Hkal \left( \z^{-2} \mathcal{D}  \uz +  \yz\right).
          \end{align}
          Using a command  $\uvec_k = -\K \widehat{\xvec}_{k+1|k}$
          \begin{align}
            \uz = -\K \left(\I - \Ad \left( \I - \Hkal \Cd
              \right)\right)^{-1} \left( \z^{-2} \mathcal{D}  \uz + \Ad \Hkal \yz\right).
          \end{align}
         Grouping terms, one gets
          \begin{equation}
            \uz = - \left( \I + \K \Lambda_{p} \z^{-2} \Lkal \mathcal{D} \right)^{-1} \K
            \Lambda_{p} \Ad \Hkal \yz,
          \end{equation}
  where 
  \begin{equation}
    \Lambda_{p} = \left(\I -  \z^{-1}\Ad \left( \I - \Hkal \Cd
              \right)\right)^{-1} .
  \end{equation}
 The controller transfer function is finally written as
  \begin{equation}
    \label{eq:Kalman_predictor_TFdemo}
    \mathbf{C}_p(\z) =  - \left( \I + \K \Lambda_{p} \z^{-2} \Lkal \mathcal{D} \right)^{-1} \K
            \Lambda_{p} \Ad \Hkal  .
  \end{equation}

The general case with an arbitrary delay and  $\Bd \neq \0$ one gets
  \begin{equation}
    \mathbf{C}_p(\z) =  - \left[ \I + \K \Lambda_{p} \left(\z^{-d} \Lkal
      \mathcal{D} + \Bd \right)\right]^{-1} \K
            \Lambda_{p} \Ad \Hkal  .
  \end{equation}


For a command $\uvec_k = -\K
  \widehat{\xvec}_{k|k}$, following an analogous reasoning 
  \begin{align}
    \widehat{\xz}_{e} & 
% = \z^{-1} \Ad \widehat{\xz}_{k|k} + \z^{-1} \Bd
%     \uz_k + \Hkal \left( \yz_k - \Cd \left[ \z^{-1}\left( \Ad
%           \widehat{\xz}_{k|k} + \Bd \uz_k\right) \right] \right) \\
%     & = \z^{-1} \Ad \widehat{\xz}_{k|k} + \z^{-1} \Bd
%     \uz_k + \Hkal \left( \yz_k - \z^{-1} \Cd \left[\left( \Ad
%           \widehat{\xz}_{k|k} + \Bd \uz_k\right) \right] \right) \\
%     & = \z^{-1} \left( \Ad - \Hkal \Cd \Ad \right) \widehat{\xz}_{k|k} +
%     \z^{-1} \left( \Bd - \Hkal \Cd \Bd \right) \uz_k + \Hkal \yz_k \\
%     & 
    = \left(\I - \z^{-1}\Ad + \z^{-1} \Hkal \Cd \Ad \right)^{-1}
    \Ad \Hkal \left( \z^{-2} \mathcal{D}  \uz +  \yz\right),
  \end{align}       
  yielding
  \begin{align}
    \uz = -\K \left(\I - \z^{-1}\Ad + \z^{-1} \Hkal \Cd \Ad \right)^{-1}
    \Ad \Hkal \left( \z^{-2} \mathcal{D}  \uz +  \yz\right)
%\left(\I - \z^{-1} \Ad + \z^{-1} \Hkal \Cd \Ad \right)^{-1} \left[ \z^{-1} \left( \I - \Hkal \Cd \right)\Bd \uz + \Hkal  \right]\yz.
  \end{align}
  Grouping terms
 \begin{equation}
    \mathbf{C}_e(\z)  = - \left(\I + \K \Lambda_{e} \z^{-2}\Hkal \mathcal{D}\right)^{-1} \K \Lambda_{e} \Hkal,
  \end{equation}
where 
  \begin{equation}
    \Lambda_{e} = \left(\I - \z^{-1}\Ad + \z^{-1} \Hkal \Cd \Ad \right)^{-1}.
  \end{equation}

The general case when  $\Bd \neq \0$ and arbitrary delay equates to
 \begin{equation}
    \mathbf{C}_e(\z)  = - \left\{\I + \K \Lambda_{e} \left[z^{-1}
        (\I+\Hkal\Cd) \Bd + \z^{-d}\Hkal\mathcal{D}\right] \right\}^{-1} \K \Lambda_{e} \Hkal,
  \end{equation}
which reduces to the previous case if  $\Bd = \0$ and $d=2$.
 $\qed$
\end{demo}

\subsection{Script to use }
Use function \verb+LQG_controller_TF.m+ to compute above functions. An
exemple is included in the function headed to compare the TFs to
matlab built-in Z-transform TFs. All cases work perfectly well.





\bibliographystyle{apalike}% falpha
{\bibliography{references}}

\end{document}

