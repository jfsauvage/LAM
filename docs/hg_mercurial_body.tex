\section{Basic use of Mercurial / Hg}

It has been noted that the biggest obstacle for adoption of VCS in
academic environment is the need to study large amount of new
commands for a VCS. The development process of scientific software
is usually less complicated than the conventional software.
Therefore, only a few basic commands are really needed for simple
yet effective control of the development process:

\begin{itemize}
\item
  adding new files to the repository (\verb!hg add!);

\item
  removing files from the repository (\verb!hg rm!);

\item
  checking the state of the repository (\verb!hg st!);

\item
  committing to the repository (\verb!hg ci -m "commit message"!);

\item
  pulling the changes from another repository / web web-based hosting
  service\\
  (\verb!hg pull https://user@bitbucket.org/owner/project!);

\item
  pushing the changes to another repository / web web-based hosting
  service\\ (\verb!hg push /home/user/backups/currentproj!);

\item
  updating the files in the repository, especially after pulling from
  another repo (\verb!hg update!);

\end{itemize}
Before that, you need either make a new repository (use
\verb!hg init! command to initialize the repository) or
\textbackslash{}textit\{clone\} an existing one
(\verb!hg cone /home/user/projects/repo!).

There are many web-based services for software projects that use
version control systems; such web services can be used for
synchronisation of repository by other researchers. As an example,
\emph{www.bitbucket.org} can be mentioned for the Mercurial DVCS.

\section{Branches in Mercurial}

If you to develop a feature, but you don't want to see it in the
mainline yet - you may want to make a branch. You can do a
\href{http://stevelosh.com/blog/2009/08/a-guide-to-branching-in-mercurial/}{branch by just cloning the repository},
but this is
\href{http://solovyov.net/blog/2011/bzr-hate-and-hate/}{not a Bazaar}
- Mercurial can do better.

\subsection{Named branches in HG/Mercurial}

Mercurial supports named branches so the branch name is a property
of the changeset (see
\href{http://mercurial.selenic.com/wiki/NamedBranches}{NamedBranches}).
If no branch name was set, Mercurial assigns the branch name
\textbf{``default''}.

The command \textbf{\texttt{hg branches}} lists all branch names
existing in a repository:

\begin{quote}
default 2635:e04d7f3658e1

\end{quote}
\begin{quote}
wfs\_lenslets\_sim 2612:9b85ce2a4a39

\end{quote}
The command \textbf{\texttt{hg branch}} may be used to set a branch
name, which will be used for subsequent commits:

\begin{quote}
hg branch am\_a\_new\_branch

\end{quote}
will start the new branch \verb!am_a_new_branch! and all the
commits will be made in this new branch.
\href{http://mercurial.selenic.com/wiki/Branch}{Mercurial branch name is an attribute associated with a changeset.}
A new branch name can be started in the middle of a development
line, not necessarily at a diverging point. For example:

\begin{quote}
hg branch branch--1 \# start a new branch name

\end{quote}
\begin{quote}
\ldots{} modify something in the repository

\end{quote}
\begin{quote}
hg commit -m ``branch--1'' \# the changeset has branch name
``branch--1''

\end{quote}
\begin{quote}
hg branch branch--2 \# start another branch name

\end{quote}
\begin{quote}
\ldots{} modify something in the repository

\end{quote}
\begin{quote}
hg commit -m ``branch--2'' \# the changeset has branch name
``branch 2''

\end{quote}
\subsubsection{Find What Branch You're On}

Calling hg branch without a name shows the current branch name of
the working directory. Calling \verb!hg branch! after a
\verb!hg init! outputs ``default'', the (reserved) name of the
default branch:

\begin{quote}
hg init

\end{quote}
\begin{quote}
hg branch

\end{quote}
\begin{quote}
default

\end{quote}
\subsubsection{Create a Branch}

To begin a new branch, set the branch name of the working directory
and then commit it:

\begin{quote}
hg branch \emph{newfeature}

\end{quote}
\begin{quote}
marked working directory as branch \emph{newfeature}

\end{quote}
\begin{quote}
hg branch

\end{quote}
\begin{quote}
\emph{newfeature}

\end{quote}
From this point on, all committed changesets will be associated
with the supplied branch name \emph{newfeature}.

\subsubsection{Create a Branch From an Older Revision}

\href{http://stackoverflow.com/questions/13549931/create-a-new-branch-at-a-certain-revision}{To get named branch BRANCHNAME},
update your working copy to the revision in question and then
creating the new branch:

\begin{quote}
hg update -r 500

\end{quote}
\begin{quote}
hg branch BRANCHNAME

\end{quote}
\begin{quote}
hg commit -m `made a new branch from revision 500'

\end{quote}
Commit \emph{is a must}, because the branch will not exist in the
repository until the next commit!

\subsubsection{Switch Between Branches}

Switch among branches using the \verb!hg update! command:

\begin{quote}
hg update -C main

\end{quote}
\begin{quote}
hg update -C newfeature

\end{quote}
By providing a branch name, \verb!hg update! will update your
working copy to the tip on this branch.

\begin{quote}
\textbf{Note: the -C option discards local changes, so be careful before using this option.}

\end{quote}
\subsubsection{Merge Branches}

\href{http://stackoverflow.com/questions/3227988/closing-hg-branches}{I've finished working on a feature branch feature-x}.
One way is to just leave merged feature branches open (and
inactive):

\begin{verbatim}
hg up default
hg merge feature-x
hg ci -m merge
\end{verbatim}
This is simpler, but it leaves an open branch. If you want merge
with \verb!default! \textbf{and close the branch}
\verb!BRANCHNAME!, do the following:

\begin{verbatim}
   hg up BRANCHNAME
   hg commit --close-branch -m "Closing the branch"
   hg up default
   hg merge BRANCHNAME
\end{verbatim}
If you just want to close the branch, do:

\begin{verbatim}
  hg commit --close-branch
\end{verbatim}
should be enough to mark a branch close. More in
\href{http://stackoverflow.com/questions/2237222/how-to-correctly-close-a-feature-branch-in-mercurial?rq=1}{this excellent post}.

Bookmarks vs Named branches

Since Mercurial 1.8 \emph{bookmarks} have become a core feature of
Mercurial. Bookmarks are more convenient for branching than named
branches. See also this question:
\href{http://stackoverflow.com/questions/1780778/mercurial-branching-and-bookmarks}{Mercurial branching and bookmarks}

\href{http://stackoverflow.com/questions/1780778/mercurial-branching-and-bookmarks?lq=1}{Bookmarks are tags}
that move forward automatically to subsequent changes, leaving no
mark on the changesets that previously had that bookmark pointing
toward them. Named branches, on the other hand, are indelible marks
that are part of a changeset. Multiple heads can be on the same
branch, but only one head at a time can be pointed to by the same
bookmark. Named branches are pushed/pulled from repo to repo, and
bookmarks don't travel.
\href{http://stevelosh.com/blog/2009/08/a-guide-to-branching-in-mercurial/}{More info about \emph{concepts} behind Mercurial bookmarks and branches.}

Is it possible to reopen a closed branch in Mercurial?

\href{http://stackoverflow.com/questions/4099345/is-it-possible-to-reopen-a-closed-branch-in-mercurial}{Yes, sure.}
You can just \verb!hg update! to the closed branch then do another
\verb!hg commit! and it will automatically reopen.

The \verb!closed! flag is just used to filter out closed branches
from \verb!hg branches! and \verb!hg heads! unless you use the
\verb!--closed! option - it doesn't prevent you from using the
branches.

Graphical view of commits and merges

The \verb!hg serve! command that starts its own webserver:

\begin{verbatim}
hg serve
listening at http://dot.dot:8000/ (bound to *:8000)
\end{verbatim}
That just point your favourite browser to
\textbf{http://dot.dot:8000/} and we can see a web server for
Mercurial!

\subsubsection{Delete / Pruning branches}

There are several ways of managing branches that you no longer plan
on working with:

\begin{enumerate}
\item
  Closing branches
\item
  No-Op Merges
\item
  Using clone
\item
  Using strip
\end{enumerate}
\href{http://mercurial.selenic.com/wiki/PruningDeadBranches}{More info about pruning the branches.}

\subsection{How can I recover a removed file in Mercurial?}

An
\href{http://stackoverflow.com/questions/2175427/how-can-i-recover-a-removed-file-in-mercurial-if-at-all}{excellent question}
indeed!
\href{http://stackoverflow.com/a/3174414}{Here is the solution.}

\textbf{First}, use

\begin{quote}
\verb!hg grep!

\end{quote}
to find the deleted file you wish to recover. The output of this
command will show you the last revision for which the file was
present, and the path to the deleted file. Or, if you know the name
of the delete file you can find its revision easily with

\begin{quote}
\verb!hg log -r "removes('NAME.c')"!

\end{quote}
This will give you the revision in witch a file called NAME.c (in
the root) is deleted.

\textbf{Second}, run

\begin{quote}
\verb!hg revert -r <revision number> <path to deleted file>!

\end{quote}
The deleted file will now be in your working copy, ready to be
committed back into head.

