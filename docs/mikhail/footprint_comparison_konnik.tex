\documentclass[a4paper,11pt]{book}
\usepackage[utf8]{inputenc}

\usepackage{amssymb,amsfonts,amsmath,mathtext,geometry,cite,enumerate}
\geometry{inner=1cm} \geometry{outer=1cm}
\geometry{top=2cm} \geometry{bottom=2cm}

\begin{document}

\chapter{Study of modal projections of Zernike in footprint\_comparison\_konnik.m}
%> @file footprint_comparison_konnik.m
%> @brief Setting up a NGS system to study modal projections of Zernike.
%> @author Mikhail Konnik
%> @date   12 May 2015

\paragraph{The goal:}
Set up a NGS (can be NTAO or MCAO) system and do some simulations with it.

Study the modal projections of Zernike defined in the metapupils (either at h=0 or at an altitude). 
For the time being we have two functions to compute those projections: \textbf{tel.footprintProjection} and 
\textbf{AnalyticalSmallFootprintExpansion.m} 

Both functions compute the same result, the 
\textbf{AnalyticalSmallFootprintExpansion.m} is much faster then the \textbf{tel.footprintProjection}, which computes the projections using numerical integration instead of an analytical method. We should therefore consider adding a better coded \textbf{AnalyticalSmallFootprintExpansion} to our \verb+lam_utilities+ sort of file or  added to the zernikeStats class.



\section{Setting up NTAO system}
Use a standard setup with atmosphere and telescope objects, but also add the asterism\footnote{Asterism is a constelation of stars, current stars geometry, especially in case of NGS/LGS} to the Guide Star object:

\begin{verbatim}
%% Guide Stars
ngs = source;

dAsterism               = 1.0; % 0.5, 0.75, 1.0, 1.5, 2 diameter in arcmin
guideStarWavelength     = photometry.R;
guideStarMagnitude      = 14;%[14,14.5,15,15.5,16,16.5,17]; % lower later
\end{verbatim}

Here we evaluate the Zernike functions on a telescope's pupil
of size tel.D and pixels nPx  and thus getting the zernProj.modes
that are just bunch of Zernike functions of maximum order maxRadialDegreeProj

\begin{verbatim}
maxRadialDegreeProj = 3;
zernModeMaxProj = zernike.nModeFromRadialOrder(maxRadialDegreeProj);
zernProj = zernike(2:zernModeMaxProj,'resolution',nPx,'pupil',tel.pupil,'D',tel.D);
\end{verbatim}

% performance estimates were done at the 1.55 m Kuiper telescope. There, an
% asterism of four natural guide stars was used to validate and estimate the
% correction achievable with a ground-layer adaptive optics system. Following from
% 
% mirror. In addition, with the scarcity of suitable asterisms of natural guide stars,
% laser guide stars would be required. With the successful demonstration of


\subsection{AnalyticalSmallFootprintExpansion}
The function \textbf{AnalyticalSmallFootprintExpansion} computes the modal projection of Zernike polynomials \textit{analytically} using Noll's indexing convention onto a smaller\footnote{Actually, the pupil of the telescope at the higher altitude is LARGER - right?!} telescope pupil, displaced by $\Delta x$ and $\Delta y$ and (possibly) rotated\footnote{In our case, the pupil is not rotated at all.}.

The major benefit of the \textbf{AnalyticalSmallFootprintExpansion} function is the speed: since there is no numerical integration, the expansion is very fast, typically 10x faster than the \textbf{tel.footprintProjection}.

The \textbf{AnalyticalSmallFootprintExpansion} has the following major steps:
\begin{enumerate}

 \item the Zernike ordering is changed from Noll conventions to ANSI, because the \textbf{TransformC} function uses ANSI:

\begin{quotation}
	A2N   =  ANSI2Noll(nMode); %% converts the ANSI order of the Zernike modes into Noll convention 
\end{quotation} 

 \item calculating telescope's diameter at an altitude of kLayer of the atmosphere:
\begin{quotation}
	D0 = tel.diameterAt(altitudes)*1000,  where:  out = obj.D + 2.*height.*tan(obj.fieldOfView/2);
\end{quotation}
here diameterAt is from telescopeAbstract, that takes into account the Field of View (FOV) and height of the layer. Then the D0 is converted to millimeters (because \textbf{TransformC} computes in these units);

\item Then for each atmospheric layer, and for each mode in that layer, we compute the projection matrix that accounts for displacement and rotation via \textbf{TransformC};

\item The matrix of projections (Proj) is computed, but for the ANSI conventions - we have to sort the columns of the projection matrix back.
\end{enumerate}

The major portion of the computation time is spent in \textbf{TransformC} (about 60\% of time), the rest is for cell2mat conversion (about 8\%)  and displacement computations (7\%).




\subsubsection{TransformC}
The \textbf{TransformC} function was proposed in the article `` Transformation of Zernike coefficients: scaled, translated, and rotated wavefronts with circular and elliptical pupils'' by Linda Lundstr\"{o}m and Peter Unsbo, published in Vol. 24, No. 3, March 2007,  J. Opt. Soc. Am. A (pages 569-577).

The algorithm presents the means to transform Zernike coefficients analytically with regard to concentric scaling, translation of pupil center, and rotation. The transformations are described both for circular and elliptical pupils. 


TransformC returns transformed Zernike coefficient set, C2, from the original set, C1, 
both in standard ANSI order, with the pupil diameter in mm as the first term. Scaling and translation is performed first and then rotation.


% Computational performance comparison: 
%     - AnalyticalSmallFootprintExpansion takes 0.1869 seconds;
%     - tel.footprintProjection takes 1.8915 seconds.
% The norm (difference) between the two methods is 6.476e-08 



\subsection{tel.footprintProjection (in telescope.m)}
Another method of Zernike functions proejctions is analytical - via  tel.footprintProjection in telescope.m file of OOMAO. Like the AnalyticalSmallFootprintExpansion, it uses the angles $\alpha$ and $\delta$ for the projection, but the projection matrix $P$ is computed 
using the function called smallFootprintExpansion (from telescope.m).

The smallFootprintExpansion(delta,largeSmallRadiusRatio) function  expands a circular portion of Zernike polynomials centered on
delta(1) and delta(2) onto another Zernike basis fitted to the circular portion. The ratio of the radius of both basis is given by \textbf{largeSmallRadiusRatio}.

The projection coefficients are found via \textbf{quad2d} function, which numerically evaluates double integral using tiled method (Quadrature
in 2D).

\end{document}